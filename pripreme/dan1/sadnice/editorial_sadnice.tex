\subsection*{Zadatak: Sadnice}
\textsf{Pripremili: Paula Vidas i Daniel Paleka}\\
\textsf{Potrebno znanje: ad-hoc, konstrukcije}\\

Broj vijenaca koji nastanu nakon reza jednak je broju presječenih komada špage
plus jedan. Ukupno ćemo koristiti $NM + N + M$ komada špage, a mogućih rezova
ima $N + M$. Svaku špagu siječe točno jedan mogući rez, pa uvijek postoji rez
koji siječe barem $\ceil{\frac{NM + N + M}{N + M}} = \ceil{\frac{NM}{N + M}} +
1$ komada špage. Pokazat ćemo da uvijek možemo postići taj odgovor.

Prva dva podzadatka su posebni slučajevi, čija ćemo (jedna od mogućih) rješenja
ilustrirati za $N = M = 6$ odnosno $N = 6, M = 12$:

\begin{verbbox}
o--o--o--o--o--o--o              o--o--o--o--o--o--o--o--o--o--o--o--o
|  |  |           |              |  |  |  |        |
o  o  o  o  o  o  o              o  o  o  o  o  o  o--o--o--o--o--o--o
         |  |  |  |                          |  |  |  |  |
o--o--o--o--o--o--o              o--o--o--o--o--o--o  o  o  o  o  o  o
|  |  |           |                                |        |  |  |  |
o  o  o  o  o  o  o              o--o--o--o--o--o--o--o--o--o--o--o--o
         |  |  |  |              |  |  |  |        |
o--o--o--o--o--o--o              o  o  o  o  o  o  o--o--o--o--o--o--o
|  |  |           |                          |  |  |  |  |
o  o  o  o  o  o  o              o--o--o--o--o--o--o  o  o  o  o  o  o
         |  |  |  |                                |        |  |  |  |
o--o--o--o--o--o--o              o--o--o--o--o--o--o--o--o--o--o--o--o
\end{verbbox}
\begin{figure}[H]
  \centering
  \theverbbox
\end{figure}

Prvo ćemo pokazati konstrukciju za parni $N$, na primjeru $N = 6, M = 17$.
Povežemo prvo sve sadnice u parnim redovima i sve sadnice u zadnjem stupcu:
\begin{verbbox}
o--o--o--o--o--o--o--o--o--o--o--o--o--o--o--o--o--o
                                                   |
o  o  o  o  o  o  o  o  o  o  o  o  o  o  o  o  o  o
                                                   |
o--o--o--o--o--o--o--o--o--o--o--o--o--o--o--o--o--o
                                                   |
o  o  o  o  o  o  o  o  o  o  o  o  o  o  o  o  o  o
                                                   |
o--o--o--o--o--o--o--o--o--o--o--o--o--o--o--o--o--o
                                                   |
o  o  o  o  o  o  o  o  o  o  o  o  o  o  o  o  o  o
                                                   |
o--o--o--o--o--o--o--o--o--o--o--o--o--o--o--o--o--o
\end{verbbox}
\begin{figure}[H]
  \centering
  \theverbbox
\end{figure}

Neka je $K = \ceil{\frac{NM}{N + M}}$. Promotrimo prvo
prvu ``prugu'', i sadnice u sredini. Prvih $K$ sadnica povežemo prema gore,
sljedećih $K$ prema dolje, a ostatak prema desno.

\begin{verbbox}
o--o--o--o--o--o--o--o--o--o--o--o--o--o--o--o--o--o
|  |  |  |  |                                      |
o  o  o  o  o  o  o  o  o  o  o--o--o--o--o--o--o--o
               |  |  |  |  |                       |
o--o--o--o--o--o--o--o--o--o--o--o--o--o--o--o--o--o
\end{verbbox}
\begin{figure}[H]
  \centering
  \theverbbox
\end{figure}

Zatim u sljedećoj pruzi radimo istu stvar, samo pomaknutu za $2K$. To jest,
počinjemo od sadnice $(1, 2K)$, a drugu koordinatu gledamo modulo $M$. Slično
nastavljamo dalje. Za promatrani primjer na kraju imamo:

\begin{verbbox}
o--o--o--o--o--o--o--o--o--o--o--o--o--o--o--o--o--o
|  |  |  |  |                                      |
o  o  o  o  o  o  o  o  o  o  o--o--o--o--o--o--o--o
               |  |  |  |  |                       |
o--o--o--o--o--o--o--o--o--o--o--o--o--o--o--o--o--o
                              |  |  |  |  |        |
o  o  o  o--o--o--o--o--o--o--o  o  o  o  o  o  o  o
|  |  |                                      |  |  |
o--o--o--o--o--o--o--o--o--o--o--o--o--o--o--o--o--o
         |  |  |  |  |                             |
o--o--o--o  o  o  o  o  o  o  o  o  o  o--o--o--o--o
                        |  |  |  |  |              |
o--o--o--o--o--o--o--o--o--o--o--o--o--o--o--o--o--o
\end{verbbox}
\begin{figure}[H]
  \centering
  \theverbbox
\end{figure}

Ako je $N$ neparan, onda u zadnjem retku srednjih $K$ sadnica ne povezujemo
prema gore, nego prema desno. Za primjer $N = 7, M = 17$ dobivamo:

\begin{verbbox}
o--o--o--o--o--o--o--o--o--o--o--o--o--o--o--o--o--o
|  |  |  |  |                                      |
o  o  o  o  o  o  o  o  o  o  o--o--o--o--o--o--o--o
               |  |  |  |  |                       |
o--o--o--o--o--o--o--o--o--o--o--o--o--o--o--o--o--o
                              |  |  |  |  |        |
o  o  o  o--o--o--o--o--o--o--o  o  o  o  o  o  o  o
|  |  |                                      |  |  |
o--o--o--o--o--o--o--o--o--o--o--o--o--o--o--o--o--o
         |  |  |  |  |                             |
o--o--o--o  o  o  o  o  o  o  o  o  o  o--o--o--o--o
                        |  |  |  |  |              |
o--o--o--o--o--o--o--o--o--o--o--o--o--o--o--o--o--o
|                                      |  |  |  |  |
o  o--o--o--o--o--o--o--o--o--o--o--o--o  o  o  o  o
\end{verbbox}
\begin{figure}[H]
  \centering
  \theverbbox
\end{figure}

Dokažimo sada da smo postigli željeno ograničenje, odnosno da svaki rez siječe
najviše $K + 1$ špaga. Vodoravne špage smo rasporedili ravnomjerno, tj. za bilo
koja dva okomita reza, broj špaga koje sijeku se razlikuje za najviše jedan.
Ako je $2K \leq M$, svaki vodoravni rez siječe točno $K + 1$ špaga, pa svaki
okomiti rez može sijeći najviše $K + 1$ špaga (u suprotnom bi prosjek bio
prevelik). Inače je $2K = M + 1$. Vodoravni rezovi tada sijeku najviše $K + 1$
dužina, a okomiti najviše $\ceil{\frac{N}{2}} + 1 \leq K + 1$ (dokaz čitatelju
za vježbu).
