\subsection*{Zadatak Redoslijed}
\textsf{Pripremio: Adrian Beker}\\
\textsf{Potrebno znanje: tournament stablo, topološko sortiranje}\\

Za početak, opisat ćemo rješenja drugog i trećeg podzadatka, u kojima su sve
boje u Davorovim potezima me\dj usobno razli\v{c}ite. Za $1 \leq i \leq N$,
neka je $P_i$ skup poteza čiji interval prekriva $i$-ti metar daske te neka
$f_i$ označava njegovu boju, odnosno neka je $f_i = 0$ ako je on neobojan.
Ukoliko je $f_i = 0$ i $P_i$ je neprazan, traženi redoslijed ne postoji, stoga
ispisujemo "NE". Tako\dj er, ako je $f_i > 0$ te $P_i$ ne sadrži potez boje
$f_i$, odgovor je "NE". U suprotnom, kako bi $i$-ti metar na kraju bio obojan
bojom $f_i$, nužan i dovoljan uvjet na redoslijed jest sljedeći: jedinstveni
potez iz $P_i$ boje $f_i$ (nazovimo ga $z_i$) dolazi poslije svih ostalih
poteza iz $P_i$. Primijetimo sada da uvjete ovog oblika možemo prikazati pomoću
usmjerenog grafa $G$ u kojemu čvorovi predstavljaju poteze, a usmjereni brid
$pq$ označava da se potez $p$ u redoslijedu nalazi prije poteza $q$. Ukoliko
$G$ ima ciklus, odgovor je "NE", a u suprotnom je traženi redoslijed moguće
naći topološkim sortiranjem ovog grafa. Naivna implementacija ovog rješenja ima
složenost $\mathcal{O}(N \cdot M)$ te je dovoljna za ostvariti sve bodove na
drugom podzadatku.

Za treći podzadatak potrebno je efikasno izgraditi spomenuti graf. U tu svrhu,
izgradimo tournament stablo $T$ nad nizom $f_i$, a graf $G$ proširimo čvorovima
stabla $T$ (ali ne i bridovima). Ovdje ćemo čvorove stabla $T$ poistovjećivati
s pripadajućim intervalima u nizu $f_i$. Interval svakog poteza $p$ podijelimo
na čvorove stabla $T$ (kao što to činimo u upitima na $T$), nazovimo taj skup
čvorova $C_p$. Za svaki $x \in C_p$ dodamo brid od $p$ prema $x$. Nadalje, za
svaki $i$ takav da je $f_i > 0$, neka $S_i$ označava skup čvorova stabla $T$
koji sadrže $f_i$ te neka je $y_i$ jedinstveni element u $C_{z_i} \cap S_i$.
Tada za svaki $x \in S_i \setminus \{y_i\}$ dodamo brid od $x$ prema $z_i$. Ako
za neki $j$ vrijedi $z_i \neq z_j$ i $y_i = y_j$, odgovor je "NE", a u
suprotnom dodamo bridove od svih poteza $p \neq z_i$ takvih da $y_i \in C_p$
prema $z_i$. Nije teško vidjeti da su valjani redoslijedi inducirani upravo
topološkim poretcima ovako izgra\dj enog grafa $G$. Budući da svaki od skupova
$C_p$, $S_i$ ima veličinu $\mathcal{O}(\log N)$, graf $G$ ima $\mathcal{O}(N +
M)$ čvorova i $\mathcal{O}((N + M)\log N)$ bridova, stoga opisano rješenje
ostvaruje sve bodove na trećem podzadatku.
\\\\
Iako nije jasno kako modificirati ovaj pristup da radi u općenitom slučaju,
potpuno rješenje zadatka koristit će neke slične ideje. Najprije za svaki $1
\leq i \leq N$ provjerimo da vrijedi $f_i > 0$ ako i samo ako se $f_i$ nalazi u
uniji intervala svih poteza -- ako taj uvjet nije ispunjen, odmah znamo da je
odgovor "NE". Dalje, traženi redoslijed pohlepno gradimo unatrag. Reći ćemo da
je neki potez \textit{dobar} ako još nije iskorišten, a trenutno se u nizu $f_i$
na njegovom intervalu pojavljuje samo njegova boja (i eventualno nule). Nije
teško tzv.\ \textit{exchange argumentom} dokazati da je sljedeći algoritam točan:
\\\\
Dok nisu svi potezi iskorišteni ponavljaj:
\begin{itemize}
\item Ako ne postoji dobar potez, odgovor je "NE";
\item Inače uzmi bilo koji dobar potez $p$, postavi sve vrijednost u nizu $f_i$
  na njegovom intervalu na $0$, te stavi $p$ na početak redoslijeda.
\end{itemize}

\noindent Primijetimo da se postavljanje elemenata na intervalu na $0$ lako
svodi na postavljanje jednog elementa na $0$ jer je svaki element potrebno
najviše jednom postaviti na $0$ (npr.\ ne-nul elemente možemo držati u
\textit{setu}). Naivna implementacija ovog algoritma ima složenost $\mathcal{O}(N
\cdot M)$ te je dovoljna za četvrti podzadatak.

Za sve bodove, preostaje efikasno održavati dobre poteze. \iffalse Odsad ćemo
\textit{bojom} zvati ne-nul vrijednost u nizu $f_i$. \fi Poteze koji su trenutno
dobri držat ćemo u redu (\textit{queueu}) $Q$. Nad nizom $f_i$ izgradimo
tournament stablo čiji svaki čvor pamti minimalnu i maksimalnu boju na svojem
intervalu (odnosno redom $\infty$, $-\infty$ ako takva boja ne postoji),
nazovimo ih $mini$ i $maks$. Tijekom algoritma, za svaki čvor razlikujemo tri
faze, ovisno o tome vrijedi li $mini < maks$, $mini = maks$ ili $mini > maks$,
odnosno redom pojavljuju li se barem dvije, točno jedna ili niti jedna boja na
tom intervalu.

Kao i u rješenju trećeg podzadatka, na početku interval svakog poteza $p$
podijelimo na čvorove u stablu te označimo dobiveni skup čvorova s $C_p$.
Tako\dj er, održavamo brojač koji broji za koliko čvorova iz skupa $C_p$
vrijedi $mini < maks$ ili $mini = maks \neq c_p$ ($c_p$ je boja poteza $p$).
Kada vrijednost tog brojača padne na $0$, potez postaje dobar i stavljamo ga u
red $Q$. Osvježavanje vrijednosti $mini$ i $maks$ u tournamentu radimo na
uobičajen način, a odgovarajuće brojače nije teško osvježiti na samom početku
te prilikom prijelaza izme\dj u faza. Ukupna je složenost $\mathcal{O}((N +
M)\log N)$. Za implementacijske detalje pogledajte službene kodove.
