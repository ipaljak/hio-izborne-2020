%%%%%%%%%%%%%%%%%%%%%%%%%%%%%%%%%%%%%%%%%%%%%%%%%%%%%%%%%%%%%%%%%%%%%%
% Problem statement
\begin{statement}[
  problempoints=100,
  timelimit=4 sekunde,
  memorylimit=512 MiB,
]{Redoslijed}

Mali Davor jednoga je dana upalio televizor i vidio kako je jedan gospodin
nacrtao prekrasan portret. ,,Kakav \textit{super talent}!'', pomislio je
Davor te odmah zgrabio kantice s bojama i svoj najdraži kist, otrčao u
dvorište te se bacio na posao.

U dvorištu je pronašao i dasku dugačku $N$ metara koju će koristiti umjesto
platna. Potom je $M$ puta umočio svoj kist u kanticu neke boje $c$ te ga je
povukao od $a$-tog do $b$-tog metra daske, obojivši taj segment bojom $c$.
Također, svakoga je puta na zaseban papirić zapisao kojom je bojom obojio
koji dio daske.

Remek-djelo je završeno, Davor je presretan, a sada još samo treba napraviti
hrpu kopija koje će izložiti po svim svjetskim galerijama. Baš dobro što je
svaki potez kistom zapisao na pap\dots

Što!? Puhnuo je vjetar i papirići su se izmiješali! Davor je slomljen, pomozite
mu odrediti kojim je redoslijedom mogao povlačiti poteze kistom tako da na
kraju dobije svoje remek-djelo ili zaključite da takav redoslijed ne postoji.
U tom je slučaju vjetar najvjerojatnije predaleko otpuhao neki papirić ili je
Davor napravio pogrešku prilikom zapisivanja.

%%%%%%%%%%%%%%%%%%%%%%%%%%%%%%%%%%%%%%%%%%%%%%%%%%%%%%%%%%%%%%%%%%%%%%
% Input
\subsection*{Ulazni podaci}
U prvom su retku prirodni brojevi $N$ i $M$ iz teksta zadatka.

U $i$-tom od idućih $M$ redaka nalaze se tri prirodna broja $a_i$, $b_i$ $(1
\le a_i \le b_i \le N)$ i $c_i$ $(1 \le c_i \le 500\,000)$ koji označavaju da
je Davor napravio potez kistom kojim je obojio dio daske od $a_i$-tog do
$b_i$-tog metra (uključivo) u boju $c_i$.

U posljednjem se retku nalazi $N$ cijelih brojeva pri čemu $i$-ti broj označava
boju kojom je obojen $i$-ti metar daske. Neobojeni dio daske označavamo brojem
$0$.

%%%%%%%%%%%%%%%%%%%%%%%%%%%%%%%%%%%%%%%%%%%%%%%%%%%%%%%%%%%%%%%%%%%%%%
% Output
\subsection*{Izlazni podaci}
U prvi redak potrebno je ispisati riječ \texttt{"DA"} ako je moguće primijeniti
Davorove poteze kistom u nekom poretku tako da konačan produkt odgovara obojenoj
dasci iz ulaza. U protivnom, potrebno je ispisati riječ \texttt{"NE"}.

Također, ako ste ispisali \texttt{"DA"}, u idućem je retku potrebno ispisati
$M$ brojeva koji označavaju kojim je redoslijedom potrebno primijeniti Davorove
poteze. Pritom nam $i$-ti ispisani broj (označimo ga s $p_i$) govori da $i$-ti
potez kistom treba odgovarati $p_i$-tom potezu navedenom u ulaznim podacima. 
Ako postoji više rješenja, ispišite bilo koje. 
%%%%%%%%%%%%%%%%%%%%%%%%%%%%%%%%%%%%%%%%%%%%%%%%%%%%%%%%%%%%%%%%%%%%%%
% Scoring
\subsection*{Bodovanje}
{\renewcommand{\arraystretch}{1.4}
  \setlength{\tabcolsep}{6pt}
  \begin{tabular}{ccl}
 Podzadatak & Broj bodova & Ograničenja \\ \midrule
  1 & 5 & $1 \le N, M \le 9$ \\
  2 & 10 & $1 \le N, M \le 5\,000$, svaki potez kistom koristit će jedinstvenu boju. \\
  3 & 25 & $1 \le N, M \le 500\,000$, svaki potez kistom koristit će jedinstvenu boju. \\
  4 & 12 & $1 \le N, M \le 5\,000$ \\
  5 & 16 & $1 \le N, M \le 500\,000$, $1 \le c_i \le 5$\\
  6 & 32 & $1 \le N, M \le 500\,000$ \\
\end{tabular}}

%%%%%%%%%%%%%%%%%%%%%%%%%%%%%%%%%%%%%%%%%%%%%%%%%%%%%%%%%%%%%%%%%%%%%%
% Examples
\subsection*{Probni primjeri}
\begin{tabularx}{\textwidth}{X'X'X}
\sampleinputs{test/redoslijed.dummy.in.1}{test/redoslijed.dummy.out.1} &
\sampleinputs{test/redoslijed.dummy.in.2}{test/redoslijed.dummy.out.2} &
\sampleinputs{test/redoslijed.dummy.in.3}{test/redoslijed.dummy.out.3}
\end{tabularx}

\textbf{Pojašnjenje prvog probnog primjera:}
\begin{itemize}
  \item Na početku je daska neobojena, odnosno njeno stanje je $(0, 0, 0, 0, 0, 0)$.
  \item Najprije od $1.$ do $4.$ metra bojimo bojom $7$ i dobivamo $(7, 7, 7, 7, 0, 0)$.
  \item Zatim od $4.$ do $6.$ metra bojimo bojom $6$ i dobivamo $(7, 7, 7, 6, 6, 6)$.
  \item Potom od $1.$ do $3.$ metra bojimo bojom $2$ i dobivamo $(2, 2, 2, 6, 6, 6)$.
  \item Onda od $3.$ do $5.$ metra bojimo bojom $5$ i dobivamo $(2, 2, 5, 5, 5, 6)$.
  \item Konačno prvi metar bojimo bojom $6$ i dobivamo $(6, 2, 5, 5, 5, 6)$.
\end{itemize}

%%%%%%%%%%%%%%%%%%%%%%%%%%%%%%%%%%%%%%%%%%%%%%%%%%%%%%%%%%%%%%%%%%%%%%
% We're done
\end{statement}

%%% Local Variables:
%%% mode: latex
%%% mode: flyspell
%%% ispell-local-dictionary: "croatian"
%%% TeX-master: "../hio.tex"
%%% End:
