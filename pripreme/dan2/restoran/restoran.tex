%%%%%%%%%%%%%%%%%%%%%%%%%%%%%%%%%%%%%%%%%%%%%%%%%%%%%%%%%%%%%%%%%%%%%%
% Problem statement
\begin{statement}[
  problempoints=100,
  timelimit=1 sekunda,
  memorylimit=512 MiB,
]{Restoran}

Sovjetski Savez, odnosno Savez Sovjetskih Socijalističkih Republika (SSSR)
nekadašnja je država koja je nastala 1917. godine nakon Oktobarske revolucije,
a raspala se 1991. godine. Popularne internetske šale nerijetko prezentiraju
život stanovnika Sovjetskoga Saveza oskudicom, neimaštinom i slabom
zastupljenošću proizvoda na tržištu. Tekst ovog zadatka, kao i sam zadatak,
inspiraciju pronalazi u takvim stereotipnim šalama te nema veze sa stvarnošću.

Ispred ulaza u sovjetski restoran stoji $N$ ljudi koji čekaju da se restoran
otvori. Ljudi su označeni prirodnim brojevima od $1$ do $N$ onim redoslijedom
kojim su se skupljali ispred restorana. Na jelovniku je samo jedna opcija,
tzv. \textit{specijalitet kuće}, jaja na oko. U restoranu ne radi kuhar, već
gosti moraju sami sebi spremiti svoje jelo. Također, u restoranu se nalazi
samo jedna tava, tako da \textbf{u nekom trenutku hranu može spremati najviše
jedan gost.} Dodatno, u restoranu se nalazi samo jedna vilica i samo jedan
nož, stoga, u \textbf{nekom trenutku hranu može jesti najviše jedan gost.}
Srećom, u restoranu se nalazi beskonačno mnogo tanjura pa gost koji završava
s pripremom hrane može gotovu hranu prebaciti na tanjur i pričekati da se
oslobode vilica i nož.

Za svakog gosta je poznato koliko mu je vremena potrebno da spremi hranu
i koliko mu je vremena da tu hranu pojede. Vaš je zadatak odrediti koliko
je najmanje vremena potrebno da se svih $N$ gostiju najede ako odluče spremati
i jesti hranu optimalnim redoslijedom.

Međutim, to nije sve, prije nego što se restoran otvorio, dogodilo se $K$
ključnih događaja oblika:

\begin{itemize}
  \item \texttt{DOLAZI a b} -- došao je novi gost koji hranu može spremiti za
        $a$ minuta, a pojesti za $b$ minuta. Novopridošli gost označen je
        najmanjim prirodnim brojem kojim nije označen niti jedan od dosadašnjih
        gostiju.
  \item \texttt{ODLAZI x} -- otišao je gost koji je $x$-ti po redu došao ispred
        restorana.
  \item \texttt{POREDAK} -- gosti su nestrpljivi i žele saznati optimalan redoslijed
        spremanja i jedenja hrane kojim će se svi najesti u najmanjem mogućem vremenu.
\end{itemize}

Prije prvog događaja potrebno je ispisati koliko je najmanje vremena potrebno
da se $N$ gostiju najede. Za svaki događaj tipa \texttt{DOLAZI} ili
\texttt{ODLAZI}, potrebno je ispisati koliko je minimalno vremena potrebno da
se najede onaj skup gostiju koji se nakon tog događaja nalazi ispred
restorana. Konačno, nakon svakog događaja tipa \texttt{POREDAK} potrebno je
ispisati kojim je redoslijedom optimalno spremati i jesti hranu da se u
najmanjem mogućem vremenu najede onaj skup ljudi koji se trenutno nalazi
ispred restorana.

%%%%%%%%%%%%%%%%%%%%%%%%%%%%%%%%%%%%%%%%%%%%%%%%%%%%%%%%%%%%%%%%%%%%%%
% Input
\subsection*{Ulazni podaci}
U prvom su retku prirodni brojevi $N$ i $K$ iz teksta zadatka.

U $i$-tom od sljedećih $N$ redaka nalaze po dva prirodna broja $a_i$ i $b_i$
$(1 \le a_i, b_i \le 10^9$ koji označavaju da gost s oznakom $i$ može hranu
spremiti za $a_i$ minuta, a pojesti za $b_i$ minuta.

U sljedećih se $K$ redaka nalazi po jedan ključan događaj u formatu kakav
je opisan u tekstu zadatka. Možete pretpostaviti da su događaji međusobno
konzistentni. Odnosno, nikad neće otići gost koji još nije niti došao. Također,
ispred restorana u svakom će se trenutku nalaziti barem jedan gost.

%%%%%%%%%%%%%%%%%%%%%%%%%%%%%%%%%%%%%%%%%%%%%%%%%%%%%%%%%%%%%%%%%%%%%%
% Output
\clearpage
\subsection*{Izlazni podaci}
U prvi je redak potrebno ispisati koliko je minimalno vremena potrebno da se
početnih $N$ gostiju najede.

Ako je $i$-ti događaj tipa \texttt{DOLAZI} ili \texttt{ODLAZI}, tada je u
$(i+1)$-om retku potrebno ispisati koliko je minimalno vremena potrebno da se
najede skup gostiju koji se nakon $i$-tog događaj nalazi ispred restorana.

Ako je $i$-ti događaj tipa \texttt{POREDAK}, tada je u $(i+1)$-om retku potrebno
ispisati $2x$ brojeva, pri čemu $x$ označava broj gostiju koji se trenutno
nalaze ispred restorana. Prvih $x$ brojeva treba sadržavati oznake gostiju onim
redom kojim će slagati hranu, dok posljednjih $x$ brojeva treba sadržavati
oznake gostiju onim redom kojim će jesti hranu.

%%%%%%%%%%%%%%%%%%%%%%%%%%%%%%%%%%%%%%%%%%%%%%%%%%%%%%%%%%%%%%%%%%%%%%
% Scoring
\subsection*{Bodovanje}
{\renewcommand{\arraystretch}{1.4}
  \setlength{\tabcolsep}{6pt}
  \begin{tabular}{ccl}
 Podzadatak & Broj bodova & Ograničenja \\ \midrule
  1 & 5 & $1 \le N \le 9$, $K = 1$, jedini događaj je tipa \texttt{POREDAK} \\
  2 & 13 & $1 \le N \le 20$, $K = 1$, jedini događaj je tipa \texttt{POREDAK} \\
  3 & 21 & $1 \le N \le 200\,000$, $K = 1$, jedini događaj je tipa \texttt{POREDAK} \\
  4 & 29 & $1 \le N, K \le 200\,000$, ne postoji događaj tipa \texttt{POREDAK} \\
  5 & 32 & $1 \le N, K \le 200\,000$, događaj tipa \texttt{POREDAK} dogodit će se najviše $10$ puta.  \\
\end{tabular}}

%%%%%%%%%%%%%%%%%%%%%%%%%%%%%%%%%%%%%%%%%%%%%%%%%%%%%%%%%%%%%%%%%%%%%%
% Examples
\subsection*{Probni primjeri}
\begin{tabularx}{\textwidth}{X'X}
\sampleinputs{test/restoran.dummy.in.1}{test/restoran.dummy.out.1} &
\sampleinputs{test/restoran.dummy.in.2}{test/restoran.dummy.out.2}
\end{tabularx}

\textbf{Pojašnjenje prvog probnog primjera:}

Gost s oznakom $1$ započinje sa slaganjem obroka i završava
u prvoj minuti. Zatim ta ista osoba započinje objed, a gost s oznakom $2$
započinje sa slaganjem svog obroka. U trećoj minuti gost s oznakom $2$ završava
sa slaganjem svog obroka, ali gost s oznakom $1$ još nije pojeo pa ovaj mora
čekati sve do četvrte minute. Konačno, u četvrtoj minuti gost oznakom $2$
započinje svoj obrok kojeg završava u sedmoj minuti.


%%%%%%%%%%%%%%%%%%%%%%%%%%%%%%%%%%%%%%%%%%%%%%%%%%%%%%%%%%%%%%%%%%%%%%
% We're done
\end{statement}

%%% Local Variables:
%%% mode: latex
%%% mode: flyspell
%%% ispell-local-dictionary: "croatian"
%%% TeX-master: "../hio.tex"
%%% End:
