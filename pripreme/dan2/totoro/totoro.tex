%%%%%%%%%%%%%%%%%%%%%%%%%%%%%%%%%%%%%%%%%%%%%%%%%%%%%%%%%%%%%%%%%%%%%%
% Problem statement
\begin{statement}[
  problempoints=100,
  timelimit=1 sekunda,
  memorylimit=512 MiB,
]{Totoro}

``\textit{Tonari no To to ro Totoro, To to ro Totoro}'' 

Totoro ima $K$ permutacija $p_1, \ldots, p_K$ duljine $N$.

Neka je $S$ skup svih permutacija koje je moguće
dobiti komponiranjem konačnog broja permutacija $p_i$.
(Dozvoljeno je ponavljati permutacije.)
Jasno je da je $S$ konačan skup.

Zanima nas \textbf{prosječan broj inverzija permutacija u skupu $S$}.
(Broj inverzija $\mathcal{I}(\pi)$ permutacije $\pi$ 
jednak je broju uređenih parova
brojeva $1 \le i < j \le N$ za koje je $\pi(i) > \pi(j)$.)

Formalno, zanima nas $\frac{1}{|S|} \sum_{\pi \in S} \mathcal{I} (\pi)$.

Može se dokazati da je odgovor moguće napisati
kao skraćeni razlomak $A/B$, gdje $B$ nije djeljiv s $10^9 + 7$. 
Ispišite $AB^{-1}$ modulo $10^9 + 7$.
%%%%%%%%%%%%%%%%%%%%%%%%%%%%%%%%%%%%%%%%%%%%%%%%%%%%%%%%%%%%%%%%%%%%%%
% Input
\subsection*{Ulazni podaci}
U prvom su retku prirodni brojevi $K$ i $N$ iz teksta zadatka.

U $i$-tom od sljedećih $K$ redaka nalazi se permutacija $p_i$, 
prikazana kao niz od $N$ različitih brojeva od $1$ do $N$, odvojenih razmakom.

%%%%%%%%%%%%%%%%%%%%%%%%%%%%%%%%%%%%%%%%%%%%%%%%%%%%%%%%%%%%%%%%%%%%%%
% Output
\subsection*{Izlazni podaci}
U jedinom retku ulaza ispišite odgovor modulo $1 000 000 007$. 


%%%%%%%%%%%%%%%%%%%%%%%%%%%%%%%%%%%%%%%%%%%%%%%%%%%%%%%%%%%%%%%%%%%%%%
% Scoring
\subsection*{Bodovanje}
{\renewcommand{\arraystretch}{1.4}
  \setlength{\tabcolsep}{6pt}
  \begin{tabular}{ccl}
 Podzadatak & Broj bodova & Ograničenja \\ \midrule
  1 & 6 & $ 1 \le K \le 10, 1 \le N \le 9$ \\
  2 & 8 & $ K = 1 $, $1 \le N \le 2\,500$ \\
  3 & 22 & $ K = 1$, $1 \le N \le 2\,500$ \\
  4 & 64 & $1 \le K \le 10$, $1 \le N \le 2\,500$ \\
\end{tabular}}

%%%%%%%%%%%%%%%%%%%%%%%%%%%%%%%%%%%%%%%%%%%%%%%%%%%%%%%%%%%%%%%%%%%%%%
% Examples
\subsection*{Probni primjeri}
\begin{tabularx}{\textwidth}{X'X'X}
\sampleinputs{test/totoro.dummy.in.1}{test/totoro.dummy.out.1} &
\sampleinputs{test/totoro.dummy.in.2}{test/totoro.dummy.out.2} &
\sampleinputs{test/totoro.dummy.in.3}{test/totoro.dummy.out.3}
\end{tabularx}

\textbf{Pojašnjenje prvog probnog primjera:}
Primijetimo da je $S = \left\{ (2, 3, 1), (3, 1, 2), (1, 2, 3)\right\}$.
Prva permutacija ima dvije inverzije, druga isto dvije inverzije, a zadnja
je identiteta i nema inverzija. Zato je prosječan broj inverzija $\frac{4}{3}$,
što odgovara ispisanom broju modulo $10^9 + 7$.

\textbf{Pojašnjenje drugog probnog primjera:}
Može se provjeriti da je $S$ jednak skupu svih permutacija skupa 
$\left\{1, 2, 3, 4, 5 \right\}$. 

%%%%%%%%%%%%%%%%%%%%%%%%%%%%%%%%%%%%%%%%%%%%%%%%%%%%%%%%%%%%%%%%%%%%%%
% We're done
\end{statement}

%%% Local Variables:
%%% mode: latex
%%% mode: flyspell
%%% ispell-local-dictionary: "croatian"
%%% TeX-master: "../hio.tex"
%%% End:
