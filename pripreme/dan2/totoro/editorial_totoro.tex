\newcommand\EE{\mathbb E}
\newcommand\PP{\mathbb P}
\newcommand\II{\mathcal I}

\subsection*{Zadatak Totoro}
\textsf{Pripremili: Stjepan Požgaj i Daniel Paleka}\\
\textsf{Potrebno znanje: matematika, pretraživanje u dubinu (DFS), 
linearnost očekivanja, teorija grupa ili Markovljevi lanci, ad hoc }

Prvo definirajmo notaciju.
Neka $\mathcal{I}(\pi)$ označava broj inverzija permutacije $\pi$.
Neka za neku tvrdnju $T$ oznaka $[T]$ označava \textit{boolean} vrijednost
te tvrdnje: $1$ ako je točna, $0$ ako je netočna. (Ta notacija naziva se 
Iverson bracket.) Kompoziciju permutacija $\tau$ i $\pi$ označavat ćemo s
$\tau \pi$.

Skup $S$ je standardan primjer \textit{grupe}: skupa na kojem je definirana
neka asocijativna operacija (ovdje kompozicija permutacija), te.
Ovdje nećemo ulaziti u dokaz da je definicijom skupa $S$ zadana podgrupa
grupe permutacija, jer čitatelj mora to jednom raspisati sam.
Jedina stvar koja na prvu nije jasna jest da možemo dobiti inverze
svake zadane permutacije, ali to je istina standardnim 
Za više, pogledate izvrsni Groups chapter iz Napkina:
{https://web.evanchen.cc/napkin.html}

Permutacije $p_i$ iz ulaza zovu se \textit{generatori} grupe.

Za prvu parcijalu, dovoljno je izračunati grupu $S$ (koja nije 
pretjerano velika) i prebrojiti inverzije u svakoj
od dobivenih permutacija. Ako je slučajno presporo, 
primijetimo da je prosječan broj inverzija
jednak $\frac{1}{2} \binom{n}{2}$
u slučaju da je grupa $S$ jednaka grupi svih permutacija $S_N$,
pa je naivno rješenje moguće ubrzati za konstantni faktor.

Za $K = 1$, dana grupa $S$ je konačna \textit{ciklička}, 
tj. oblika je $\{ 1, p, p^2, p^3, \ldots, p^m \}$ za neki $m$.
(Ovdje s $p^t$ označavamo permutaciju $p$ primijenjenu $t$ puta.)
Poznato je i lako za dokazati da je $m$ točno najmanji
zajednički višekratnik veličina svih ciklusa permutacije.

Za drugu parcijalu, vrijedi $m = N$, pa je dovoljno u $O(N \log N)$
prebrojati inverzije svake permutacije u $S$. 
Ako permutacija nije ciklus, tada $m$ može biti jako velik
(npr. ciklusi veličina prvih tridesetak prostih brojeva), 
pa moramo napraviti nešto pametnije.

Za treću parcijalu i puno rješenje, trebamo koncept \textit{linearnosti
očekivanja}. Sve što slijedi može se izreći i elementarno,
ali je vjerojatnosna terminologija puno prirodnija.

Ako nasumce (s uniformnom vjerojatnošću) biramo permutaciju $\pi$ iz
$S$, tada je izraz $\frac{1}{|S|} \II(\pi)$ prosječna vrijednost
ili \textit{očekivanje} $\EE\mathcal{I}(\pi)$ broja inverzija $\mathcal{I}$.

Po definiciji vrijedi
\begin{equation*}
  \II(\pi) = \sum\limits_{1 \le i < j \le N} [ \pi(i) > \pi(j) ]. 
\end{equation*}
Koristeći da očekivanje možemo rastaviti po pribrojnicima, dobivamo
\begin{align*}
  \EE \II(\pi)  &= \sum\limits_{1 \le i < j \le N} \EE [ \pi(i) > \pi(j) ] \\
                &= \sum\limits_{1 \le i < j \le N} \PP [ \pi(i) > \pi(j) ]. 
\end{align*}

Zato je dovoljno izračunati vjerojatnost da vrijedi $\pi(i) > \pi(j)$
kada uzimamo uniformno nasumičnu permutaciju $\pi$ iz $S$.

U slučaju kada je $K = 1$, zadatak se može lijepo matematički riješiti
u složenosti $\sum_{C_1, C_2 \text{ciklusi}} |C_1| \cdot |C_2| 
= O(N^2)$, jer možemo računati $\PP[\pi(i) < \pi(j)$ za
sve $i \in C_1, j \in C_2$ istovremeno. Za detalje pogledajte službeni kod.
rješenja za $K = 1$.

Kada je $K > 1$, moramo zaboraviti na rješenje za $K = 1$, jer je jako
teško analizirati strukturu grupe s više generatora.

Ključna ideja je promatrati graf $G_{\text{parovi}})$ 
od $N^2$ čvorova $(i, j) : 1 \le i, j \le N$,
gdje su povezani vrhovi $(i, j)$ i $(p_k(i), p_k(j))$ za svaki $1 \le k \le N$.
Ako primijenimo permutaciju $\pi$ iz $S$, jasno je da to možemo
interpretirati kao ``vrh $(i, j)$ ide u vrh $(\pi(i), \pi(j)$''.

Vrh $(i, j)$ neka permutacija $\pi$ iz $S$ 
može poslati samo u vrhove pripadajuće povezane komponente.
Ako bi znali distribuciju 
Ključna je sljedeća obzervacija:

\textbf{Tvrdnja:} Uniformno nasumična permutacija $\pi$ iz $S$ šalje
vrh $(i, j)$ u uniformno nasumičan vrh u njegovoj povezanoj komponenti.
\textit{Prvi dokaz:} Elementaran, bit će u finalnoj verziji editoriala.
\textit{Drugi dokaz:} Primijetimo da svaki vrh ima stupanj $K$,
tj. graf je \textit{regularan}.
Promatrajmo Markovljev lanac $M$ na vrhovima grafa,
tj. slučajnu šetnju koja svaki brid bira s jednakom vjerojatnošću.
Lako se dokaže da \textit{stacionarna distribucija} (jedinstvena
vjerojatnost na vrhovima grafa koja se ne mijenja u koraku lanca $M$)
svakom vrhu neke povezane komponente pridružuje istu vjerojatnost,
jer je graf regularan. 

Promotrimo \emph{Cayleyjev graf} grupe $S$ generirane permutacijama 
$p_1, \ldots, p_K$, gdje su vrhovi elementi $\pi \in S$, a bridovi
između permutacija $\pi$ i $p_k \pi$ za svaki $1 \le k \le K$.
Graf je također regularan stupnja $K$, pa ako definiramo sličnu slučajnu
šetnju $M_S$, stacionarna distribucija je uniformna po svim elementima
grupe $S$.

Sada samo koristimo da slučajna šetnja iz nekog vrha po vjerojatnosti
konvergira u stacionarnu distribuciju, kako u lancu $M$, tako i u
lancu $M_S$.
\footnote{Postoje manji problemi s periodičnošću, ali se oni lako riješe 
takozvanim \textit{lazy} bridovima -- ne ulazimo u tehnikalije.}

Slučajne šetnje na lancima $M$ i $M_S$ možemo bijektivno upariti, jer svaki
korak u oba lanca odgovara nekoj ulaznoj permutaciji $p_k$.

Zato vrijedi
\begin{align*}
  \text{uniformna permutacija } \pi \in S 
  &\equiv \text{slučajna šetnja permutacijama } p_k \\
  &\equiv \text{stacionarna distribucija na komponentama grafa }
  G_\text{parovi}.
\end{align*}

Stoga, dovoljno je za svaku povezanu komponentu u $G_\text{parovi}$ 
izračunati broj čvorova $(a, b)$ takvih da je $a > b$ (\textit{invertiranih}
čvorova),
jer se čvor $(i, j)$ šalje u svaki čvor komponente s jednakom vjerojatnošću.
Tražena vjerojatnost dobiva se dijeljenjem broja invertiranih čvorova
s veličinom komponente.
Komponente je lako izračunati u složenosti $O(N^2 K)$ pretraživanjem u dubinu.

