%%%%%%%%%%%%%%%%%%%%%%%%%%%%%%%%%%%%%%%%%%%%%%%%%%%%%%%%%%%%%%%%%%%%%%
% Problem statement
\begin{statement}[
  problempoints=100,
  timelimit=1 sekunda,
  memorylimit=512 MiB,
]{Autoritet}

Gospodin Malnar globalno je priznat kao autoritet za mnoge stvari. Primjerice,
autoritet je po pitanju kvalitete suhomesnatih proizvoda, ekološkog uzgoja
ljutih papričica na balkonskim prostorima, degustacije sokova na bazi grožđa
i mnogih drugih stvari.  U ovom ćemo se zadatku baviti problemom koji ga
trenutno mori te ćemo istražiti kako će gospodin Malnar svoj problem riješiti
koristeći neosporan autoritet u avioindustriji.

Naime, gospodin Malnar je ove godine imao zakazane letove u Singapur i Moskvu.
Već je rezervirao avionske karte, odabrao prostran smještaj i proučio
najbolje \textit{wellness \& spa} destinacije. Nažalost, uslijed epidemiološke
krize, putovanja su otkazana. Sav shrvan i zabrinut, odmah je krenuo
proučavati redove letenja i opće stanje avioindustrije te primijetio da
\textbf{svijet više nije povezan}. ,,\textit{To tako ne može, moram pod hitno
spasiti svijet!}'', pomislio je gospodin Malnar i bacio se na posao.

Na svijetu postoji $N$ zračnih luka i $M$ zračnih linija. Zračne luke
označavamo prirodnim brojevima od $1$ do $N$, a svaka zračna linija spaja
neke dvije različite zračne luke, što znači da avioni mogu u oba smjera
putovati između te dvije luke. U normalnim je okolnostima bilo moguće iz
svake zračne luke proputovati do bilo koje druge zračne luke koristeći jednu
ili više zračnih linija, odnosno, svijet je bio povezan. Gospodin Malnar će
svijet ponovo povezati sa svega nekoliko telefonskih poziva. Svaki poziv bit
će upućen nekoj zračnoj luci, neki će pozivi možda više puta biti upućeni
istoj luci, a teći će otprilike ovako:

\textbf{Predstavnik zračne luke:} Dobar dan! Dobili ste zračnu luku, kako vam
mogu pomoći?

\textbf{Gospodin Malnar:} Dobar dan, gospodin Malnar pri telefonu. Primijetio
sam da vaše zračne linije nemaju smisla i da trebate napraviti potpuno
suprotnu stvar. Odnosno, neka skup $A$ sadrži zračne luke s kojima ste
dikretno spojeni zračnom linijom, a neka skup $B$ sadrži sve ostale zračne
luke. Želim da ukinete sve zračne linije koje spajaju vašu luku i luke iz
skupa $A$ te da uvedete zračne linije koje će spajati vašu luku i luke iz
skupa $B$. Ja sad imam nekog posla pa moram ići, vi napravite kako sam rekao.

\textbf{Predstavnik zračne luke:} Ispričavamo se na propustu, postupit ćemo
  kako ste rekli.

Vaš je zadatak odrediti koji je najmanji broj telefonskih poziva koje gospodin
Malnar mora obaviti kako bi ponovo spojio svijet. Također, odredite na koliko
je različitih načina mogao obavljati pozive, a da i dalje broj obavljenih
poziva bude minimalan. Broj načina potrebno je ispisati modulo $10^9 + 7$.
Moguće je dokazati da, koristeći dovoljno telefonskih poziva, gospodin Malnar
uvijek može spasiti svijet.

%%%%%%%%%%%%%%%%%%%%%%%%%%%%%%%%%%%%%%%%%%%%%%%%%%%%%%%%%%%%%%%%%%%%%%
% Input
\subsection*{Ulazni podaci}
U prvom su retku prirodni brojevi $N$ i $M$ iz teksta zadatka.

U $i$-tom od idućih $M$ redaka nalaze dva prirodna broja $a_i$ i $b_i$ $(1 \le
a_i, b_i \le N, a_i \ne b_i)$ koji označavaju da postoji zračna linija između
zračnih luka s oznakama $a_i$ i $b_i$. Neće postojati dvije zračne linije
koje spajaju isti par zračnih luka.

%%%%%%%%%%%%%%%%%%%%%%%%%%%%%%%%%%%%%%%%%%%%%%%%%%%%%%%%%%%%%%%%%%%%%%
% Output
\subsection*{Izlazni podaci}
U prvom retku ispišite traženi najmanji broj telefonskih poziva iz teksta
zadatka.

U drugom retku ispišite traženi broj načina iz teksta zadatka modulo $10^9 + 7$.

%%%%%%%%%%%%%%%%%%%%%%%%%%%%%%%%%%%%%%%%%%%%%%%%%%%%%%%%%%%%%%%%%%%%%%
% Scoring
\clearpage
\subsection*{Bodovanje}
Rješenja koja na nekom test podatku ispišu točan prvi redak i pogrešan drugi
redak (ili ga uopće ne ispišu), osvojit će $15\%$ bodova predviđenih za taj
test podatak.

Broj bodova nekog podzadatka jednak je najmanjem broju bodova koje vaše rješenje
ostvaruje na nekom od test podataka tog podzadatka.

{\renewcommand{\arraystretch}{1.4}
  \setlength{\tabcolsep}{6pt}
  \begin{tabular}{ccl}
 Podzadatak & Broj bodova & Ograničenja \\ \midrule
  1 & 5 & $1 \leq N \leq 18$ \\
  2 & 9 & $1 \leq N, M \leq 300$ \\
  3 & 16 & $1 \leq N, M \leq 3\,000$ \\
  4 & 70 & $1 \leq N, M \leq 500\,000$ \\
\end{tabular}}

%%%%%%%%%%%%%%%%%%%%%%%%%%%%%%%%%%%%%%%%%%%%%%%%%%%%%%%%%%%%%%%%%%%%%%
% Examples
\subsection*{Probni primjeri}
\begin{tabularx}{\textwidth}{X'X'X}
\sampleinputs{test/autoritet.dummy.in.1}{test/autoritet.dummy.out.1} &
\sampleinputs{test/autoritet.dummy.in.2}{test/autoritet.dummy.out.2} &
\sampleinputs{test/autoritet.dummy.in.3}{test/autoritet.dummy.out.3}
\end{tabularx}

\textbf{Pojašnjenje prvog probnog primjera:}
Svijet je već povezan, stoga gospodin Malnar ne treba obaviti niti jedan poziv.

\textbf{Pojašnjenje drugog probnog primjera:}
Sljedeći su nizovi poziva najkraći me\dj u onima koji svijet čine povezanim: $(1, 4)$, $(4, 1)$, $(2, 3)$, $(3, 2)$.
%%%%%%%%%%%%%%%%%%%%%%%%%%%%%%%%%%%%%%%%%%%%%%%%%%%%%%%%%%%%%%%%%%%%%%
% We're done
\end{statement}

%%% Local Variables:
%%% mode: latex
%%% mode: flyspell
%%% ispell-local-dictionary: "croatian"
%%% TeX-master: "../hio.tex"
%%% End:
