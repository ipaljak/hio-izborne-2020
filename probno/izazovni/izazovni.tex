%%%%%%%%%%%%%%%%%%%%%%%%%%%%%%%%%%%%%%%%%%%%%%%%%%%%%%%%%%%%%%%%%%%%%%
% Problem statement
\begin{statement}[
  problempoints=100,
  timelimit=1 sekunda,
  memorylimit=512 MiB,
]{Izazovni}

Ovo je priča o dvojici neprijatelja, Miletu i Miroslavu, koji se, usprkos
mnogim sličnostima, nikako ne mogu složiti kada raspravljaju o nekoj temi.
Mileta je to toliko počelo živcirati da je odlučio izazvati Miroslava da
riješi jedan zadatak.  Nakon pola sata, uspjeli su se složiti da ga obojica
ne znaju riješiti.  Znate li vi? Prihvaćate li izazov?

Pronađite niz koji se sastoji od $N$ nula i jediica takav da je njegov najdulji
palindromski podniz što manji.

Podnizom nekog niza smatramo bilo kakav niz uzastopnih elemenata tog niza.
Primjerice, \texttt{101}, \texttt{11} i \texttt{1101} podnizovi su niza
\texttt{1101}, dok \texttt{00} i \texttt{010} to nisu. Za podnizove koji se
jednako čitaju slijeva nadesno, kao i zdesna nalijevo, kažemo da su
palindromski podnizovi. Primjerice, \texttt{1} i \texttt{010} palindromski su
podnizovi niza \texttt{00101}, dok \texttt{01} i \texttt{0101} to nisu.

%%%%%%%%%%%%%%%%%%%%%%%%%%%%%%%%%%%%%%%%%%%%%%%%%%%%%%%%%%%%%%%%%%%%%%
% Input
\subsection*{Ulazni podaci}
U prvom je retku prirodan broj $N$ iz teksta zadatka.

%%%%%%%%%%%%%%%%%%%%%%%%%%%%%%%%%%%%%%%%%%%%%%%%%%%%%%%%%%%%%%%%%%%%%%
% Output
\subsection*{Izlazni podaci}
U jedini redak ispišite traženi niz iz teksta zadatka bez znaka razmaka
između susjednih elemenata.

%%%%%%%%%%%%%%%%%%%%%%%%%%%%%%%%%%%%%%%%%%%%%%%%%%%%%%%%%%%%%%%%%%%%%%
% Scoring
\subsection*{Bodovanje}
Rješenja koja na nekom test podatku ispišu ispravno formatiran niz od $N$ nula
i jedinica osvojit će $X\cdot(1+A-B)^{-0.4}$ bodova pri čemu je $A$ nadulji
palindromski podniz vašeg rješenja, $B$ je najdulji palindromski podniz
optimalnog rješenja, a $X$ je broj bodova predviđenih za taj test podatak.

Broj bodova nekog podzadatka jednak je najmanjem broju bodova koje vaše rješenje
ostvaruje na nekom od test podataka tog podzadatka.

{\renewcommand{\arraystretch}{1.4}
  \setlength{\tabcolsep}{6pt}
  \begin{tabular}{ccl}
 Podzadatak & Broj bodova & Ograničenja \\ \midrule
  1 & ?? & $1 \le N \le 10$ \\
  2 & ?? & $1 \le N \le 1\,000\,000$\\
\end{tabular}}

%%%%%%%%%%%%%%%%%%%%%%%%%%%%%%%%%%%%%%%%%%%%%%%%%%%%%%%%%%%%%%%%%%%%%%
% Examples
\subsection*{Probni primjeri}
\begin{tabularx}{\textwidth}{X'X}
\sampleinputs{test/izazovni.dummy.in.1}{test/izazovni.dummy.out.1} &
\sampleinputs{test/izazovni.dummy.in.2}{test/izazovni.dummy.out.2}
\end{tabularx}

\textbf{Pojašnjenje probnih primjera:} probni primjeri prikazuju dva
različita izlaza za jednake ulazne podatke. Prvi probni primjer predstavlja
optimalno rješenje, odnosno optimalna duljina najduljeg palindromskog podniza
u binarnom nizu duljine $3$ iznosi $2$, a jedan niz koji to postiže jest upravo
\texttt{001}. Drugi probni primjer predstavlja suboptimalno rješenje koje
bi postiglo $75.79\%$ predviđenih bodova na ovom test podatku.

%%%%%%%%%%%%%%%%%%%%%%%%%%%%%%%%%%%%%%%%%%%%%%%%%%%%%%%%%%%%%%%%%%%%%%
% We're done
\end{statement}

%%% Local Variables:
%%% mode: latex
%%% mode: flyspell
%%% ispell-local-dictionary: "croatian"
%%% TeX-master: "../hio.tex"
%%% End:
