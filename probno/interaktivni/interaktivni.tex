%%%%%%%%%%%%%%%%%%%%%%%%%%%%%%%%%%%%%%%%%%%%%%%%%%%%%%%%%%%%%%%%%%%%%%
% Problem statement
\begin{statement}[
  problempoints=100,
  timelimit=5 sekundi,
  memorylimit=512 MiB,
]{Interaktivni}

Ovo je priča o dvojici lopova, Ivici i Milanu, koji pokušavaju ukrasti skriveno
blago u jednom drevnom južnoameričkom hramu. Trenutno se u unutrašnjosti
hrama nalazi samo Ivica, dok Milan vani čuva stražu. Naravno, naši dragi
kriminalci koriste voki-toki za komunikaciju. Ivica je u nekom trenutku ušao
u ogromnu sobu popločenu kvadratnim poljima u kojoj je ugledao nešto nalik
blagu te pojurio u tom smjeru. Nažalost, stao je na posebno polje (zamku)
koje je uzrokovalo da se ulazna vrata zatvore, neka polja postanu neprohodna,
a prostoriju ispuni tama. Ivica je odmah zapalio baklju, ali njena svjetlost
nije daleko dopirala.  Odnosno, Ivica je samo mogao vidjeti svoju neposrednu
okolinu.

Odlučio je javiti Milanu što se dogodilo te ga upitati za pomoć. Nakon kraćeg
razgovora, odlučili su da će Ivica govoriti što vidi oko sebe, a Milan će mu
reći u kojem smjeru da se pomakne. Uz malo sreće i algoritamskog znanja, Ivica
će ukrasti blago i pobjeći iz sobe. Također, u drevnim spisima piše da je za
izlazak iz sobe potrebno znati kolika je najkraća udaljenost između blaga i
zamke na koju je Ivica stao (njegove početne pozicije).

Milan je ipak za svaki slučaj zamolio vas za pomoć. Pomozite mu i napišite
program koji simulira interakciju s Ivicom te ga uspješno navodi kroz hram.

%%%%%%%%%%%%%%%%%%%%%%%%%%%%%%%%%%%%%%%%%%%%%%%%%%%%%%%%%%%%%%%%%%%%%%
% Input
\subsection*{Interakcija}
Ovo je interaktivni zadatak. Vaš program treba uspostaviti komunikaciju sa
programom izrađenim od strane organizatora koji simulira ponašanje Ivice iz
teksta zadatka. Dakako, vaš program treba simulirati Milanovo ponašanje kojim
će dvojac ukrasti blago i odrediti najkraću udaljenost od blaga do Ivičine
početne pozicije. Najkraća udaljenost između neka dva polja u hramu jest
najmanji broj koraka koje Ivica treba napraviti da bi s prvog polja došao do
drugog polja, ako se pritom smije kretati u četiri osnovna smjera (gore,
dolje, lijevo i desno).

Interakciju započinje Ivica koji Milanu javlja što se nalazi u neposrednoj
okolini njegove početne pozicije. Odnosno, vaš program najprije treba sa
standardnog ulaza pročitati matricu znakova dimenzija $3\times3$ koja
predstavlja Ivičinu okolinu. Svaki znak matrice predstavlja jedno polje,
pritom znak \texttt{'.'} označava prohodno polje, znak \texttt{'\#'} označava
neprohodno polje, a znak \texttt{'B'} označava prohodno polje s blagom. Polje
matrice koje se nalazi u drugom retku i drugom stupcu uvijek će biti prohodno
i ono predstavlja Ivičinu trenutnu poziciju.

Vaš program može tada narediti Ivici da se pomakne u jednom od četiri osnovna
smjera tako da na standardni izlaz ispiše redak oblika ``\texttt{POMAK X}'',
gdje je \texttt{X} jedan od znakova \texttt{'R'}, \texttt{'D'}, \texttt{'L'}
ili \texttt{'U'} koji redom označavaju da se Ivica treba pomaknuti udesno,
prema dolje, ulijevo ili prema gore. Nakon svake ispisane naredbe, vaš
program treba napraviti \textit{flush} izlaza. Također, nakon svakog pomaka,
vaš program sa standardnog ulaza treba pročitati Ivičinu okolinu nakon
napravljenog pomaka, u istom formatu kako je opisano u prethodnom odlomku.

Kada vaš program odredi kolika je najkraća udljenost između blaga i Ivičine
početne pozicije, na standardni izlaz treba ispisati redak oblika
``\texttt{\frenchspacing! D}'', gdje \texttt{D} predstavlja traženu
udaljenost. Nakon toga, vaš program treba napraviti \textit{flush} izlaza i
završiti izvođenje.

\textbf{Napomena:} Putem sustava za evaluaciju možete preuzeti primjere
izvornih kodova koji na ispravan način obavljaju interakciju, uključujući
\textit{flush} izlaza.

%%%%%%%%%%%%%%%%%%%%%%%%%%%%%%%%%%%%%%%%%%%%%%%%%%%%%%%%%%%%%%%%%%%%%%
% Scoring
\subsection*{Bodovanje}
Testni primjer smatramo ispravno riješenim ako vaš program izda najviše $Q$
naredbi tipa ``\texttt{POMAK}'', nikada ne pokuša Ivicu natjerati da stane na
neprohodno polje, barem jednom posjeti polje na kojem se nalazi blago te
ispravno odredi najkraću udaljenost između Ivičine početne pozicije i polja s
blagom. Znak $N$ u donjoj tablici predstavlja broj različitih prohodnih polja
koja Ivica može posjetiti sa svoje početne pozicije krečući se u četiri osnovna
smjera.

{\renewcommand{\arraystretch}{1.4}
  \setlength{\tabcolsep}{6pt}
  \begin{tabular}{ccl}
 Podzadatak & Broj bodova & Ograničenja \\ \midrule
  1 & 50 & $1 \le N \le 16$, $Q = 200\,000$\\
  2 & 50 & $1 \le N \le 100\,000$, $Q = 200\,000$\\
\end{tabular}}

%%%%%%%%%%%%%%%%%%%%%%%%%%%%%%%%%%%%%%%%%%%%%%%%%%%%%%%%%%%%%%%%%%%%%%
% Examples
\subsection*{Primjer interakcije}
U probnom je primjeru Ivica zarobljen u hramu oblika:

\begin{centering}
\texttt{\#\#\#\#\#\#}\\
\texttt{\#.I\#.\#}\\
\texttt{\#.\#\#.\#}\\
\texttt{\#.B..\#}\\
\texttt{\#\#\#\#\#\#}\\
\end{centering}

gdje je znakom \texttt{I} označena njegova početna pozicija, a znakom \texttt{B}
pozicija polja s blagom.\\

{\renewcommand{\arraystretch}{1}
  \setlength{\tabcolsep}{6pt}
  \begin{tabular}{lcl}
    Izlaz & Ulaz & Napomena \\ \midrule
      & \texttt{\#\#\#} & \\
      & \texttt{..\#} & Okolina Ivičine početne pozicije. \\
      & \texttt{.\#\#} & \\
    \texttt{POMAK L} & & Milan izdaje naredbu Ivici da se pomakne ulijevo. \\
      & \texttt{\#\#\#} & \\
      & \texttt{\#..} & Okolina Ivičine trenutne pozicije. \\
      & \texttt{\#.\#} & \\
    \texttt{POMAK D} & & Milan izdaje naredbu Ivici da se pomakne prema dolje. \\
      & \texttt{\#..} & \\
      & \texttt{\#.\#} & Okolina Ivičine trenutne pozicije. \\
      & \texttt{\#.B} & \\
    \texttt{POMAK D} & & Milan izdaje naredbu Ivici da se pomakne prema dolje. \\
      & \texttt{\#.\#} & \\
      & \texttt{\#.B} & Okolina Ivičine trenutne pozicije. \\
      & \texttt{\#\#\#} & \\
    \texttt{POMAK R} & & Milan izdaje naredbu Ivici da se pomakne udesno. \\
      & \texttt{.\#\#} & \\
      & \texttt{.B.} & Ivica se dokopao tajnog blaga. \\
      & \texttt{\#\#\#} & \\
    \texttt{\frenchspacing! 4} & & Milan ispravno zaključuje da je najkraća udaljenost jednaka $4$. \\
\end{tabular}}

%%%%%%%%%%%%%%%%%%%%%%%%%%%%%%%%%%%%%%%%%%%%%%%%%%%%%%%%%%%%%%%%%%%%%%
% We're done
\end{statement}

%%% Local Variables:
%%% mode: latex
%%% mode: flyspell
%%% ispell-local-dictionary: "croatian"
%%% TeX-master: "../hio.tex"
%%% End:
