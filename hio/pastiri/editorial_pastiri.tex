\subsection*{Zadatak: Pastiri}
\textsf{Pripremili: Daniel Paleka, Adrian Beker}\\
\textsf{Potrebno znanje: grafovi, pretraga u širinu/dubinu (BFS/DFS)}
\\\\
Neka je $V$ skup čvorova stabla. U opisu rješenja poistovjećivat ćemo ovcu/pastira i čvor u kojem se nalazi. Primijetimo najprije da zadatak možemo promatrati kao instancu takozvanog \emph{set cover} problema. Zaista, ako svakom čvoru $v \in V$ pridružimo skup $S_v$ ovaca koje čuva pastir u $v$, tada je potrebno naći najmanji podskup $P \subseteq V$ takav da je $\bigcup_{v \in P} S_v$ cijeli skup ovaca. Iako je generalna verzija ovog problema NP-potpuna, specifična struktura dotičnog slučaja omogućit će nam da ga riješimo u odgovarajućoj složenosti.

Krenimo od prvog podzadatka, odnosno slučaja kada je zadano stablo lanac. Tada pastir može čuvati samo prvu ovcu lijevo/desno od sebe, stoga se familija $\{S_v \mid v \in V\}$ sastoji od sljedećih skupova:
\begin{itemize}
    \item $\{x\}$ za svaku ovcu $x$;
    \item $\{x, y\}$ za susjedne ovce $x, y$ takve da su $x$, $y$ iste parnosti.
\end{itemize}
Nameće se sljedeći pohlepni algoritam -- pogledamo prvu ovcu, u slučaju da je druga ovca iste parnosti, postavimo pastira na pola puta izme\dj u njih, u suprotnom ga postavimo u čvor prve ovce. Nakon toga uklonimo pokrivene ovce te ponavljamo algoritam. Opisano rješenje ima složenost $\mathcal{O}(N + K)$.

U drugom podzadatku, podskupove ovaca predstavljat ćemo bitmaskama, odnosno brojevima iz skupa $\{0, 1, \ldots, 2^K-1\}$. Na početku pustimo BFS/DFS iz svake ovce te na taj način u složenosti $\mathcal{O}(K \cdot N)$ odredimo skup $S_v$ za svaki čvor $v$. Dalje zadatak rješavamo dinamičkim programiranjem. Za svaku bitmasku $mask$ neka $f(mask)$ označava minimalan broj skupova $S_v$ čija unija sadrži $mask$. Iteriramo kroz stanja $mask$ u rastućem poretku. U prijelazu iteriramo kroz sve podmaske $submask$ te ukoliko $submask$ odgovara nekom od skupova $S_v$, osvježavamo $f(mask)$ vrijednošću $f(mask \mathbin{^\wedge} submask) + 1$ (ovdje $^\wedge$ označava bitovno isklučivo ili). Nakon toga preostaje za sve podmaske $submask$ osvježiti $f(submask)$ vrijednošću $f(mask)$. Memorijska je složenost $\mathcal{O}(N + 2^K)$, a vremenska $\mathcal{O}(K \cdot N + 3^K)$ (dokaz ove standardne činjenice ostavljamo čitateljici za vježbu).

Za preostale podzadatke, ukorijenimo stablo u proizvoljnom čvoru. Za svaku ovcu $x$, njenim \emph{teritorijem} zvat ćemo skup $\{v \in V \mid x \in S_v\}$. Za dvije ovce $x, y$ reći ćemo da su \emph{prijatelji} ako njihovi teritoriji imaju neprazan presjek. Ideja je pohlepno postavljati pastira koji čuva neku ovcu i sve njene dosad nepokrivene prijatelje. To će nam omogućiti sljedeća tvrdnja:
\\\\
\textbf{Tvrdnja 1.} Za neku ovcu $x$, neka je $a(x)$ njen najviši predak koji se nalazi u njenom teritoriju. Tada $S_{a(x)}$ sadrži sve prijatelje od $x$ koji nisu dublji od $x$.
\\\\
\textit{Dokaz.} Neka je ovca $y$ prijatelj od $x$ koji nije dublji od $x$. Tada možemo pretpostaviti da se $y$ ne nalazi u podstablu od $a(x)$ jer u suprotnom je tvrdnja očita. Uzmimo čvor $z$ koji se nalazi u teritorijima od $x$ i $y$ te neka je čvor $w$ polovište puta od $x$ do $y$. Tada je 
\newpage 
$$d(z, x) = d(z, y) = d(z, w) + d(w, x) = d(z, w) + d(w, y),$$ gdje $d(u, v)$ označava udaljenost čvorova $u, v$. Tako\dj er, za bilo koju ovcu $t$ vrijedi $$d(z, w) + d(w, x) = d(z, x) \leq d(z, t) \leq d(z, w) + d(w, t),$$ odakle slijedi $d(w, x) \leq d(w, t)$, a analogno se pokazuje $d(w, y) \leq d(w, t)$. Dakle, $w$ se nalazi u presjeku teritorija od $x$ i $y$. Štoviše, budući da $y$ nije dublja od $x$, $w$ je predak od $x$, stoga se nalazi na putu od $x$ do $a(x)$. Budući da $y$ nije u podstablu od $a(x)$, vrijedi $$d(a(x), y) \leq d(w, y) = d(w, x) \leq d(a(x), x),$$ stoga pastir u $a(x)$ čuva $y$, što je i trebalo dokazati. $\qed$
\\\\
Prema Tvrdnji 1, sljedeći je algoritam ispravan:
\begin{itemize}
    \item Dok god nisu sve ovce pokrivene ponavljaj:
    \begin{itemize}
        \item Postavi pastira u $a(x)$, gdje je $x$ najdublja dosad nepokrivena ovca.
    \end{itemize}
\end{itemize}
Direktna implementacija ovog algoritma ima složenost $\mathcal{O}(N(N + K))$ te je dovoljna za ostvariti sve bodove na trećem podzadatku. Za četvrti podzadatak potrebno je ubrzati opisani algoritam. Za svaki čvor $v$, neka $dep(v)$, $dist(v)$ redom označavaju njegovu dubinu i udaljenost do najbliže ovce. Na početku možemo izračunati $dist(v)$ pomoću BFS-a iz svih ovaca. Alternativno, možemo zamisliti da smo stablu dodali \emph{dummy} čvor povezan sa svim ovcama i iz njega pustili BFS. Sljedeća obzervacija sada nam omogućuje da efikasno odredimo čvor $a(x)$ za svaku ovcu $x$:
\\\\
\textbf{Obzervacija 2.} Ako je čvor $v$ predak ovce $x$, tada je $dist(v) \leq dep(x) - dep(v)$, i jednakost vrijedi ako i samo ako je $v$ u teritoriju ovce $x$.
\\\\
Prema Obzervaciji $2$, $a(x)$ je pozicija prvog pojavljivanja maksimuma od $dist(v) + dep(v)$ po svim čvorovima $v$ na putu od korijena do $x$. Stoga je $a(x)$ za svaku ovcu $x$ moguće izračunati jednostavnim DFS-om iz korijena. Konačno, preostaje efikasno održavati najdublju dosad nepokrivenu ovcu. Ako sortiramo ovce padajuće po dubini, taj se problem svodi na održavanje pokrivenih ovaca. U tu svrhu za početni BFS promotrimo pripadajući \emph{graf najkraćih putova}. To je usmjeren graf $G$ sa skupom vrhova $V$ te bridovima $(u, v)$ gdje je $\{u, v\}$ brid stabla takav da je $dist(v) = dist(u) + 1$.
\\\\
\textbf{Obzervacija 3.} Za svaki čvor $v \in V$, $S_v$ je skup ovaca $x$ takvih da postoji put od $x$ do $v$ u $G$. 
\\\\
Kad god postavimo novog pastira, proširimo se DFS-om iz njega unatrag po grafu $G$ te pritom pazimo da obilazimo samo dosad neposjećene čvorove. Prema Obzervaciji 3, ovca je pokrivena ako i samo smo ju posjetili u DFS-u. Budući da svaki čvor posjećujemo najviše jednom, održavanje pokrivenih ovaca ima ukupno linearnu složenost.

Na kraju, ukupna je složenost $\mathcal{O}(N + K\log K)$. Primijetimo još da se faktor $\log K$ pojavljuje isključivo zbog sortiranja ovaca po dubini, što je naravno moguće izvesti i u složenosti $\mathcal{O}(N + K)$ jer dubine ne prelaze $N$. Stoga postoji rješenje i u linearnoj složenosti. Za implementacijske detalje pogledajte službene kodove.