\subsection*{Zadatak: Ulica}
\textsf{Pripremio: Marin Kišić}\\
\textsf{Potrebno znanje: petlje, provjera parnosti}

Napravit ćemo pomoćno polje \texttt{moze} i za svaki broj $x$ iz ulaza ćemo
postaviti \texttt{moze[x]=1}. Nakon toga, prebrojimo ima li više parnih ili
neparnih brojeva u u ulazu. Na kraju, ako ima više parnih, krenemo od $2$ i
idemo tako dugo dok je \texttt{moze[trenutni\_broj]=1} krećući se po parnim
brojevima. Analogno za neparne, samo počinjemo od $1$.

\subsection*{Zadatak: Datum}
\textsf{Pripremio: Karlo Franić}\\
\textsf{Potrebno znanje: provjera palindromičnosti, ad-hoc}

Za prvi podzadatak potrebno je uzeti prva dva znaka datuma i taj broj
povećavati za $1$ dok ne nađemo palindrom.

Za drugi podzadatak koristimo znanje
prvog podzadatka, uz to svaki put kada dan dođe do zadnjeg u mjesecu (4.\ i 5.\
znak), postavimo dan na $1$, a mjesec povećamo za $1$.

Za treći podzadatak potrebno
je povećavati i godinu svaki put kad dođemo do $31$.\ dana u $12$.\ mjesecu.

Za sve
bodove bilo je potrebno primijetiti da palindromičnih datuma postoji $366$. Svaki
datum ima točno jednu godinu s kojom tvori palindrom. Te datume je moguće
staviti u polje i nakon toga za svaki zadani datum iterativno po polju naći
prvi datum koji je veći od zadanog.

Vremenska složenost je $\mathcal{O}(NK)$, gdje $K$ predstavlja broj
palindromičnih datuma. Zadatak se može riješiti i u složenosti $\mathcal{O}(N
\log K)$, a to rješenje ostavljamo čitateljici za vježbu.
