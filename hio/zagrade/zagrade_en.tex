%%%%%%%%%%%%%%%%%%%%%%%%%%%%%%%%%%%%%%%%%%%%%%%%%%%%%%%%%%%%%%%%%%%%%%
% Problem statement
\begin{statement}[
  problempoints=100,
  timelimit=3 seconds,
  memorylimit=512 MiB,
]{Zagrade}

It is well known that the Central Intelligence Agency is tasked with gathering,
processing and analyzing national security information. It is also suspected
that they own quite large collections of commonly-used computer passwords and
are developing some quite sophisticated tools that are capable of compromising
password-protected computer systems.

The tables have turned, your task is to compromise the security of a CIA
server. Good luck!

Naturally, they are well aware of typical patterns that humans produce while
coming up with their passwords, so attempts such as \texttt{123456},
\texttt{password}, \texttt{1q2w3e4r} or \texttt{welcome} are futile. Luckily,
we have uncovered certain pieces of information that might be of use to you.

Namely, their master password consists of exactly $N$ characters, where $N$
is an even number. Exactly half of those characters is equal to the open
parenthesis (\texttt{'('}), while the other half is equal to the closing
parenthesis (\texttt{')'}). Additionally, instead of the usual
``\textit{forgot your password?}'' functionality, their engineers have decided
to expose an API to the forgetful administrator. Using the API, an administrator
can execute at most $Q$ queries asking ,,\textit{whether the interval of
  the password from $a$-th to the $b$-th character is mathematically valid}''.

The mathematical validity of a sequence of parentheses is defined inductively
as:

\begin{itemize}
  \item \texttt{()} is a mathematically valid sequence.
  \item If $A$ is a mathematically valid sequence, then
    \texttt{(}$A$\texttt{)} is a mathematically valid sequence as well.
  \item If both $A$ and $B$ are mathematically valid sequences, then
        $AB$ is also mathematically valid.
\end{itemize}

%%%%%%%%%%%%%%%%%%%%%%%%%%%%%%%%%%%%%%%%%%%%%%%%%%%%%%%%%%%%%%%%%%%%%%
% Input
\subsection*{Interaction}
This is an interactive task. Your program must communicate with a program made
by the organizers which simulates the functionality of a \textbf{fictitious}
insecure CIA server from the task description.

Before interaction, your program should read an even integer $N$ and an integer
$Q$ from the standard input. The meaning of both numbers is described in the
task statement.

After that, your program can send query requests by writing to the standard
output. Each query must be printed in a separate line and have the
form ``\texttt{?} \textit{a b}'', where $1 \le a \le b \le N$ holds. After
each query has been written, your program should \textbf{flush} the output
and read the \textit{answer} from the standard input. The answer is a $1$ if
the interval of the password starting from $a$-th and ending at the $b$-th
character forms a mathematically valid sequence of parentheses. Otherwise,
the answer is $0$. Your program can make at most $Q$ such queries.

After your program has deduced the secret password, it should write a line to
the standard output in the form ``\texttt{!} \textit{$x_1x_2\dots x_N$}'', where
characters $x_1$, $x_2$, \dots, $x_N$ represent the characters of the secret
password. After that, your program should \textit{flush} the output once more
and gracefully terminate its execution.

\textbf{Note:} You can download the sample source code from the judging system
that correctly interact with the CIA server, including the output flush.

%%%%%%%%%%%%%%%%%%%%%%%%%%%%%%%%%%%%%%%%%%%%%%%%%%%%%%%%%%%%%%%%%%%%%%
% Scoring
\subsection*{Scoring}
{\renewcommand{\arraystretch}{1.4}
  \setlength{\tabcolsep}{6pt}
  \begin{tabular}{ccl}
    Subtask & Score & Constraints \\ \midrule
    1 & 14 & $1 \le N \le 1\,000$, $Q = \frac{N^2}{4}$, the whole password is a
            mathematically valid sequence. \\
    2 & 7 & $1 \le N \le 1\,000$, $Q = \frac{N^2}{4}$ \\
    3 & 57 & $1 \le N \le 100\,000 $, $Q = N - 1$,
             the whole password is a mathematically valid sequence. \\
    4 & 22 & $1 \le N \le 100\,000 $, $Q = N - 1$ \\
\end{tabular}}

%%%%%%%%%%%%%%%%%%%%%%%%%%%%%%%%%%%%%%%%%%%%%%%%%%%%%%%%%%%%%%%%%%%%%%
% Examples
\subsection*{Interaction Example}
{\renewcommand{\arraystretch}{1.4}
  \setlength{\tabcolsep}{6pt}
  \begin{tabular}{lcl}
    Input & Output & Comment \\ \midrule
      & 6 9 & The secret password is \texttt{((()))} of length 6, and a program may ask at most $9$ queries. \\
    \texttt{\frenchspacing? 1 6} & \texttt{1} & The whole password is mathematically valid. \\
    \texttt{\frenchspacing? 1 2} & \texttt{0} & \texttt{((} isn't mathematically valid. \\
    \texttt{\frenchspacing? 2 4} & \texttt{0} & \texttt{(()} isn't mathematically valid. \\
    \texttt{\frenchspacing? 2 5} & \texttt{1} & \texttt{(())} is mathematically valid. \\
    \texttt{\frenchspacing? 3 4} & \texttt{1} & \texttt{()} is mathematically valid. \\
    \texttt{\frenchspacing! ((()))} & & The password is correctly deduced and CIA is compromised. \\
\end{tabular}}

%%%%%%%%%%%%%%%%%%%%%%%%%%%%%%%%%%%%%%%%%%%%%%%%%%%%%%%%%%%%%%%%%%%%%%
% We're done
\end{statement}

%%% Local Variables:
%%% mode: latex
%%% mode: flyspell
%%% ispell-local-dictionary: "croatian"
%%% TeX-master: "../hio.tex"
%%% End:
