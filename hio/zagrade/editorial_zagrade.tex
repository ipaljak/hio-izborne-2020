\subsection*{Zadatak: Zagrade}
\textsf{Pripremila: Paula Vidas}\\
\textsf{Potrebno znanje: stog ???}

Prvo ćemo opisati rješenje u slučaju da je cijela lozinka validan niz. U prvom
podzadatku je $Q = \frac{N^2}{4}$, pa možemo postaviti upit za svaki interval
parne duljine (primjetimo da nema smisla pitati za intervale neparne duljine).
Jedno moguće rješenje je sljedeće: pitamo za prva dva znaka, prva četiri znaka,
itd. sve dok ne dobijemo pozitivan odgovor. Tada znamo na kojem je mjestu
zatvorena zagrada koja je uparena s otvorenom zagradom na prvom mjestu. Sada
rekurzivno riješimo isti zadatak na intervalu između tih zagrada i na intervalu
desno od zatvorene zagrade.

U trećem podzadatku možemo postaviti $Q = N - 1$ upita. Koristit ćemo stog, koji
je na početku prazan. Idemo redom po pozicijama lozinke. Ako je stog prazan,
stavimo trenutnu poziciju na stog. Inače, postavimo upit za interval između
pozicije koja je na vrhu stoga i trenutne pozicije. Ako je odgovor pozitivan,
zagrade na tim pozicijama su uparene, te obrišemo vrh stoga. U suprotnom stavimo
trenutnu poziciju na stog.

Što kada cijela lozinka nije validan niz? Algoritam je isti, ali na kraju stog
ne mora ostati prazan. Od pozicija koje su ostale na stogu, na prvih (manjih)
pola mora biti zatvorena, a na drugih pola otvorena zagrada.
