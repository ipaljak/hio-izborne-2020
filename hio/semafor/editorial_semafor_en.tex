\subsection*{Task: Semafor}
\textsf{Prepared by: Bojan Štetić, Ivan Paljak, Daniel Paleka}\\
\textsf{Necessary skills: matrix multiplication, graph theory, binary
exponentiation}

We will solve the harder case M=2. 

We represent a state of the board by a bitmask of 10 bytes.
Call a bitmask \textit{nice} if the represented board shows a valid number.

It is well known that the vertices of the \textit{hypercube graph} 
are all bitmasks of some size (10 here), and the edges connect bitmasks
which differ in exactly one place. Let $H$ denote the adjacency matrix
of the hypercube.

Then the problem, formally, asks for the following: for each nice bitmask, 
find the number of walks of length $N$ in the hypercube, 
starting from the state $X$, such that the walk visits the set of nice bitmasks
every $K$ steps.

\textbf{Claim 1:} The number of walks of length $D$ in the hypercube, starting
in $X$ and ending in $Y$, equals the $(X, Y)$-th entry in the matrix $H^D$.

\textit{Proof:} Look into Daniel A. Spielman's "Spectral and Algebraic Graph 
Theory", brilliant stuff. For an elementary proof, it's Lemma 3 on the link: 
\\
{https://courses.grainger.illinois.edu/cs598cci/sp2020/LectureNotes/lecture1.pdf}
$\qed$

\textbf{Definition 2:} Let $B$ be the principal submatrix of $H^K$ indexed by
only the rows and columns of the nice bitmasks.

\textbf{Claim 3:} The number of walks (from $X$ to $Y$) of length $N$ in the
hypercube, where $K$ divides $N$, such that every $K$ steps the walk visits a
nice bitmask, equals the $(X, Y)$-th entry in the matrix $B^{N/K}$.

\textit{Proof:} Left as exercise. $\qed$

There are several steps in the solution. We need to calculate the matrix $B$,
then calculate $B^{N/K}$, and then multiply it with the matrix corresponding
to the last $N \mod K$ steps. We omit the third because it is very similar
to the first step, so we assume $K$ divides $N$ in the rest of the exposition.

Let us solve the second step (raising $B$ to the $N/K$-th power) first.
It is enough to use naive binary exponentiation and multiply two matrices
in the ordinary cubic complexity, because for $M=2$ we are dealing with $100
\times 100$ matrices.

The first step is tougher, because $B$ is a submatrix of $H^K$, and $H$ is a
$1024 \times 1024$ matrix for $M=2$. Naive calculation of $H^K$ passes only the
subtask $M=1$. For smaller $K$ and $M=2$, it can be done with clever dynamic
programming, and this should give the first four subtasks.

For the full problem, we need to raise the hypercube matrix to some power
faster.

\textbf{Claim 4:} The $(X, Y)$-th entry of the matrix $B$ depends only on the
number of bits in which the bitmasks $X$ and $Y$ differ.

\textit{Proof:} Without losing of generality, we can $XOR$ all vertices of the
hypercube with some fixed bitmask. Thus, we can assume $X = 0$.
It's also clear that the order of the bits doesn't matter at all.
Hence, the number of walks from $0$ to $Y$ depends only on the bitcount of $Y$.

Claim 4 shows that's it's enough to calculate the first row of the matrix $H^K$.
Depending on the exact implementation, it can pass the first five subtasks, or
even pass the whole problem. (Ask Dorijan Lendvaj for the XOR-convolution 
solution of quadratic complexity that runs faster than the official solution.)

The official solution calculates the numbers of walks starting in $X=0$ 
in a smarter way. It's enough to calculate, for every $0 \le r \le 10$,
 the number of walks starting from bitcount $0$ and ending in bitcount $r$.

The number of ways in which we can get from one bitcount $0 \le r \le 10$ to
another is given by the following $11 \times 11$ matrix:
\[
  C =
\begin{bmatrix}
  0       & 10 & 0 & 0 & \dots  & 0 & 0 & 0 \\
    1       & 0 & 9 & 0 & \dots & 0 & 0 & 0 \\
    0       & 2 & 0 & 8 & \dots & 0 & 0 & 0 \\
    \hdotsfor{8} \\
    0       & 0 & 0 & 0 & \dots & 9 & 0 & 1 \\
    0       & 0 & 0 & 0 & \dots & 0 & 10 & 0 \\
\end{bmatrix}
\]
By raising the matrix $C$ to a suitable power, we can get the number of walks of
length $K$ from bitcount $0$ to each bitcount $0 \le r \le 10$.
To get the actual number of walks in the hypercube, we just need to divide by a
suitable binomial coefficient, due to symmetry.
For implementation details, see the official source codes.

There is an even faster solution, which uses \textit{exponential generating
functions}. The number of walks of length $K$ from the bitmask $0$ to some mask
with $10-r$ bits equals the coefficient next to $\frac{x^K}{K!}$ of the
following expression:
\[
  \left(\sum_{i=0}^{\infty} \frac{x^{2i}}{(2i)!}\right)^{r}
  \left(\sum_{i=0}^{\infty} \frac{x^{2i+1}}{(2i+1)!}\right)^{10-r} \\
  = \left( \frac{e^x + e^{-x}}{2} \right)^{r}
  \left( \frac{e^x - e^{-x}}{2} \right)^{10-r}.
\]
By expanding the expression carefully, we get a solution of the complexity   
$O((5\cdot M)^2 + (5\cdot M) \log K)$, that is, entries of the $n \times n$
hypercube matrix raised to some power $K$ is possible in $O(\log n \log K)$ 
time.

For similar ideas, look into Herbert Wilf's free book
\textit{generatingfunctionology}.

