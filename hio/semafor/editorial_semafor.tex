\subsection*{Zadatak: Ulica}
\textsf{Pripremili: Bojan Štetić, Ivan Paljak, Daniel Paleka}\\
\textsf{Potrebno znanje: matrično množenje, teorija grafova, brzo potenciranje}

Kao prvo, stanje semafora predstavljamo kao bitmasku od 5 ili 10 bitova, ovisno
o podzadatku. Bitmasku koja predstavlja ispravno stanje nazovimo
\emph{dobrom}.

Svako dijete znade da \emph{graf hiperkocke} za vrhove ima sve bitmaske nekog
broja bitova (u ovom zadatku 10), a bridovima su povezani vrhovi koji se
razlikuju na točno jednom mjestu. Neka je $H$ matrica susjedstva hiperkocke.

Tada zadatak formalno glasi: pronađite broj šetnji duljine $N$ u hiperkocki 
od stanja $X$ do svake od dobrih bitmaski, takvih da šetnja svakih K koraka
stane u nekoj dobroj bitmaski.

\textbf{Tvrdnja 1.} Broj šetnji duljine $N$ u hiperkocki, s početkom u $X$ 
i završetkom u $Y$, jednak je poziciji $(X, Y)$ u matrici $H^N$.

\textit{Dokaz:} Pogledajte Spielmanov knjigu/pdf o algebarskoj teoriji grafova,
fantastična stvar. Za elementarni dokaz, pogledajte
\\
{https://courses.grainger.illinois.edu/cs598cci/sp2020/LectureNotes/lecture1.pdf}




