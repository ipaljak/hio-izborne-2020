\subsection*{Zadatak: Semafor}
\textsf{Pripremili: Bojan Štetić, Ivan Paljak, Daniel Paleka}\\
\textsf{Potrebno znanje: matrično množenje, teorija grafova, brzo potenciranje}

Kao prvo, stanje semafora predstavljamo kao bitmasku od 5 ili 10 bitova, ovisno
o podzadatku. Bitmasku koja predstavlja ispravno stanje nazovimo
\emph{dobrom}.

Svako dijete znade da \emph{graf hiperkocke} za vrhove ima sve bitmaske nekog
broja bitova (u ovom zadatku 10), a bridovima su povezani vrhovi koji se
razlikuju na točno jednom mjestu. Neka je $H$ matrica susjedstva hiperkocke.

Tada zadatak formalno glasi: pronađite broj šetnji duljine $N$ u hiperkocki 
od stanja $X$ do svake od dobrih bitmaski, takvih da šetnja svakih K koraka
stane u nekoj dobroj bitmaski.

\textbf{Tvrdnja 1.} Broj šetnji duljine $D$ u hiperkocki, s početkom u $X$ 
i završetkom u $Y$, jednak je poziciji $(X, Y)$ u matrici $H^D$.

\textit{Dokaz:} Pogledajte Spielmanov knjigu/pdf o algebarskoj teoriji grafova,
fantastična stvar. Za elementarni dokaz, pogledajte
\\
{https://courses.grainger.illinois.edu/cs598cci/sp2020/LectureNotes/lecture1.pdf}

\textbf{Definicija 2.} Neka je $B$ matrica dobivena tako što od matrice $H^K$ 
ostavimo samo retke i stupce koji odgovaraju dobrim bitmaskama.

\textbf{Tvrdnja 3.} Broj šetnji (od $X$ do $Y$) duljine $N$ u hiperkocki, gdje je $N$ djeljiv
s $K$, takvih da svakih $K$ koraka budemo u dobrom stanju, jednak je 
polju $(X, Y)$ u matrici $B^{N/K}$.

\textit{Dokaz:} Ostavljeno za vježbu.


Predstavljamo rješenje za teži slučaj, tj. $M=2$. 
Potrebno je izračunati matricu $B$, potencirati je na $\floor{N/K}$, te
pomnožiti još s ostatkom koraka.

Potencirati matricu $B$ je dovoljno učiniti naivnim putem,
jer za $M=2$ ona ima $100$ redaka i stupaca.

Međutim, potenciranje matrice $H$ (oblika $1024 \times 1024$) je presporo
naivnim putem, jer bi vrijeme izvršavanja bilo $O(1024^3 \log K)$.

Zato, treba smisliti kako pametnije potencirati matricu hiperkocke.
Uz Walsh-Hadamard pristupe, ili funkcije izvodnice, u koje u ovom trenutku
pisanja editoriala nećemo ulaziti, najlakši pristup je sljedeće:

\textbf{Tvrdnja 4:} Element $(X, Y)$ matrice $B$ ovisi samo o broju
bitova u kojima se $X$ i $Y$ razlikuju.

\textit{Dokaz:} Lagana vježbica, možemo bez smanjenja općenitosti pretpostaviti
$X=0$.

Ako uvedemo simetriju i računamo za $X=0$, dovoljno je izračunati 
broj šetnji koje počnu u bitcountu $0$ i završe u nekom bitcountu $r$, za svaki
$0 \le r \le 10$. To možemo lagano izračunati potenciranjem matrice $11 \times
11$. 
Za detalje, pogledajte kod službenog rješenja.








