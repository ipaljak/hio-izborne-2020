\subsection*{Zadatak: Semafor}
\textsf{Pripremili: Bojan Štetić, Ivan Paljak, Daniel Paleka}\\
\textsf{Potrebno znanje: matrično množenje, teorija grafova, brzo potenciranje}

Kao prvo, stanje semafora predstavljamo kao bitmasku od 5 ili 10 bitova, ovisno
o podzadatku. Bitmasku koja predstavlja ispravno stanje nazovimo
\emph{dobrom}.

Dobro poznati \emph{graf hiperkocke} za vrhove ima sve bitmaske nekog
broja bitova (u ovom zadatku 10), a bridovima su povezani vrhovi koji se
razlikuju na točno jednom mjestu. Neka je $H$ matrica susjedstva hiperkocke.

Tada zadatak formalno glasi: pronađite broj šetnji duljine $N$ u hiperkocki 
od stanja $X$ do svake od dobrih bitmaski, takvih da šetnja svakih K koraka
stane u nekoj dobroj bitmaski.

\textbf{Tvrdnja 1.} Broj šetnji duljine $D$ u hiperkocki, s početkom u $X$ 
i završetkom u $Y$, jednak je poziciji $(X, Y)$ u matrici $H^D$.

\textit{Dokaz:} Pogledajte Spielmanov knjigu/pdf o algebarskoj teoriji grafova,
fantastična stvar. Za elementarni dokaz, pogledajte Lemma 3 na linku
\\
{https://courses.grainger.illinois.edu/cs598cci/sp2020/LectureNotes/lecture1.pdf}
$\qed$

\textbf{Definicija 2.} Neka je $B$ matrica dobivena tako što od matrice $H^K$ 
ostavimo samo retke i stupce koji odgovaraju dobrim bitmaskama.

\textbf{Tvrdnja 3.} Broj šetnji (od $X$ do $Y$) duljine $N$ u hiperkocki, gdje je $N$ djeljiv
s $K$, takvih da svakih $K$ koraka budemo u dobrom stanju, jednak je 
polju $(X, Y)$ u matrici $B^{N/K}$.

\textit{Dokaz:} Ostavljeno za vježbu.$\qed$

Predstavljamo rješenje za teži slučaj, tj. $M=2$. 
Potrebno je izračunati matricu $B$, potencirati je na $\floor{N/K}$, te
pomnožiti još s ostatkom koraka. Pošto je dio vezan za ostatak koraka
nakon zadnjeg višekratnika od $K$ potpuno identičan kao
dio vezan za računanje matrice $B$, dalje opisujemo rješenje
u slučaju kada je $N$ djeljiv s $K$.

Potencirati matricu $B$ je dovoljno učiniti naivnim brzim potenciranjem,
množeći matrice u kubnoj složenosti, jer je za $M=2$ ona dimenzija $100 \times 100$.

Međutim, potenciranje matrice $H$ radi samo za $M=1$, jer
je za $M=2$ ona oblika $1024 \times 1024$.

Za dovoljno mali $K$, umjesto potenciranja matrica možemo koristiti
dinamiku, koja za neki početni vrh $X$ računa broj načina 
za doći u svaki vrh hiperkocke u nekom broju koraka. Svako
stanje dinamike utječe na $10$ drugih stanja, pa je 
to rješenje dovoljno brzo za $K <= 15$.

Međutim, ograničenja su $K \le N \le 10^{18}$, što znači 
da je potrebno pametnije potencirati matricu hiperkocke.
Ključna je sljedeća tvrdnja:

\textbf{Tvrdnja 4:} Element $(X, Y)$ matrice $B$ ovisi samo o broju
bitova u kojima se $X$ i $Y$ razlikuju.

\textit{Dokaz:} Jasno je da sve vrhove hiperkocke možemo 
bez smanjenja općenitost bitovno $XOR$-ati s nekom bitmaskom.
Zato možemo pretpostaviti $X=0$ u ovoj tvrdnji.
Također, jasno je da poredak bitova ne igra nikakvu ulogu.
Zato su svi $Y$ koji imaju isti broj jedinica ekvivalentni 
što se tiče broja šetnji od $0$ do $Y$. $\qed$


Tvrdnja 4 implicira da je dovoljno brzo izračunati prvi redak matrice $B = H^K$. 
Ovisno o implementaciji, takvo rješenje može
riješiti subtask $K \le 1500$ ili proći za sve bodove.
(Natjecatelj Dorijan Lendvaj je implementirao XOR-konvoluciju,
koja radi jako brzo na standardnim procesorskim arhitekturama.)

Službeno rješenje ima drukčiju ideju.
Ako uvedemo simetriju i računamo za $X=0$, dovoljno je izračunati 
broj šetnji koje počnu u bitcountu $0$ i završe u nekom bitcountu $r$, za svaki
$0 \le r \le 10$. 
Broj načina za doći iz nekog bitcounta u neki drugi u jednom koraku
zadan je sljedećom $11 \times 11$ matricom:
\[
  C = 
\begin{bmatrix}
  0       & 10 & 0 & 0 & \dots  & 0 & 0 & 0 \\
    1       & 0 & 9 & 0 & \dots & 0 & 0 & 0 \\
    0       & 2 & 0 & 8 & \dots & 0 & 0 & 0 \\
    \hdotsfor{8} \\
    0       & 0 & 0 & 0 & \dots & 9 & 0 & 1 \\
    0       & 0 & 0 & 0 & \dots & 0 & 10 & 0 \\
\end{bmatrix}
\]
Ako potenciramo matricu $C$, dobit ćemo broj šetnji duljine $K$ od bitcounta $0$
do svakog drugog bitcounta.
Za dovršiti rješenje, opet koristimo da je broj šetnji do svake
bitmaske nekog fiksnog bitcounta jednak, pa je ukupan broj
šetnji do zadanog bitcounta dovoljno podijeliti s 
odgovarajućim binomnim koeficijentom.

Za detalje, pogledajte izvorni kod službenog rješenja.

Postoji još brže rješenje, koje koristi \emph{eksponencijalne funkcije
izvodnice}. Ukratko, traženi broj šetnji duljine $K$ od bitcounta $0$
do neke maske s $10-r$ bitova jednak je koeficijentu
uz $\frac{x^K}{K!}$ sljedećeg izraza:
\[
  \left(\sum_{i=0}^{\infty} \frac{x^{2i}}{(2i)!}\right)^{r}
  \left(\sum_{i=0}^{\infty} \frac{x^{2i+1}}{(2i+1)!}\right)^{10-r} \\
  = \left( \frac{e^x + e^{-x}}{2} \right)^{r}
  \left( \frac{e^x - e^{-x}}{2} \right)^{10-r}.
\]
Uz pažljivo raspisivanje, moguće je rješenje broja operacija 
$O((5\cdot M)^2 + (5\cdot M) \log K)$, to jest, dobiti elemente
potencirane matrice hiperkocke
nekih dimenzija $n \times n$ je moguće u $O(\log n \log K)$.

Za slične ideje, pogledajte besplatnu knjigu
\textit{generatingfunctionology} Herberta Wilfa.








