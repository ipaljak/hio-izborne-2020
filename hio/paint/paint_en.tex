%%%%%%%%%%%%%%%%%%%%%%%%%%%%%%%%%%%%%%%%%%%%%%%%%%%%%%%%%%%%%%%%%%%%%%
% Problem statement
\begin{statement}[
  problempoints=100,
  timelimit=3 seconds,
  memorylimit=512 MiB,
]{Paint}

We will represent the drawing area of \textit{MS Paint} as a rectangular
grid of unit squares divided into $R$ rows and $S$ columns. Each square of
the grid represents a single pixel that can be colored in one of the $10^5$
different colors. When the user applies the so called \textit{bucket tool} with
color $A$ on a pixel $(r, s)$ which is colored in the color $B$, then all pixels
in the \textit{monochrome neighborhood} of pixel $(r, s)$ change their color
to $A$. Monochrome neighborhood of a pixel $(r, s)$ is a set of pixels that
are reachable by \textit{walking} from $(r, s)$ in the four general directions
(up, down, left and right) without changing the color of the pixel along the
way. Note that the pixel $(r, s)$ is itself a part of its monochrome
neighborhood.

\begin{figure}[H]
\centering
\includegraphics[width=0.8\textwidth]{img/paint_skica.png}
\end{figure}

You are given a starting image drawn in \textit{MS Paint} along with $Q$
instructions that should be executed in the given order. Each instruction tells
you on which pixel should you apply the bucket tool and with what color. Your
task is to how the image looks like after all instructions are executed.

%%%%%%%%%%%%%%%%%%%%%%%%%%%%%%%%%%%%%%%%%%%%%%%%%%%%%%%%%%%%%%%%%%%%%%
% Input
\subsection*{Input}
The first line contains integers $R$ and $S$ from the task description.

Each of the next $R$ lines contains $S$ non-negative integers less than
$100\,000$ that represent the starting image drawn in \textit{MS Paint}.
More precisely, the $j$-th number in the $i$-th row of the image represents
the color of the pixel $(i, j)$.

The next line contains an integer $Q$ from the task description.

The $i$-th of the next $Q$ lines contains integers $r_i$, $s_i$ and
$c_i$ $(1 \le r_i \le R, 1 \le s_i \le S, 0 \le c_i < 100\,000)$, which
represent the $i$-th instruction that tells you to use the bucket
tool with color $c_i$ on the pixel $(r_i, s_i)$.


%%%%%%%%%%%%%%%%%%%%%%%%%%%%%%%%%%%%%%%%%%%%%%%%%%%%%%%%%%%%%%%%%%%%%%
% Output
\subsection*{Output}
You should output the final state of the image in the same format as
it was given in the input.

%%%%%%%%%%%%%%%%%%%%%%%%%%%%%%%%%%%%%%%%%%%%%%%%%%%%%%%%%%%%%%%%%%%%%%
% Scoring
\subsection*{Scoring}
{\renewcommand{\arraystretch}{1.4}
  \setlength{\tabcolsep}{6pt}
  \begin{tabular}{ccl}
 Subtask & Score & Constraint \\ \midrule
  1 & 8 & $1 \le R \cdot S \le 10\,000$, $1 \le Q \le 10\,000$ \\
  2 & 9 & $R = 1$, $1 \le S \le 200\,000$, $1 \le Q \le 100\,000$ \\
  3 & 31 & \makecell[l] { $1 \le R \cdot S \le 200\,000$, $1 \le Q \le 100\,000$ \\
  Each pixel will in every moment be colored either in color $0$ or color $1$.} \\
  4 & 52 & $1 \le R \cdot S \le 200\,000$, $1 \le Q \le 100\,000$
\end{tabular}}

%%%%%%%%%%%%%%%%%%%%%%%%%%%%%%%%%%%%%%%%%%%%%%%%%%%%%%%%%%%%%%%%%%%%%%
% Examples
\subsection*{Examples}
\begin{tabularx}{\textwidth}{X'X'X}
\sampleinputs{test/paint.dummy.in.1}{test/paint.dummy.out.1} &
\sampleinputs{test/paint.dummy.in.2}{test/paint.dummy.out.2} &
\sampleinputs{test/paint.dummy.in.3}{test/paint.dummy.out.3}
\end{tabularx}

\textbf{Clarification of the first example:} The figure from the task
description corresponds to the input of the first example. White color
corresponds to number $0$, red color corresponds to number $1$,
blue color corresponds to number $2$, green color corresponds to number
$3$ and yellow color corresponds to number $4$.

%%%%%%%%%%%%%%%%%%%%%%%%%%%%%%%%%%%%%%%%%%%%%%%%%%%%%%%%%%%%%%%%%%%%%%
% We're done
\end{statement}

%%% Local Variables:
%%% mode: latex
%%% mode: flyspell
%%% ispell-local-dictionary: "croatian"
%%% TeX-master: "../hio.tex"
%%% End:
