\subsection*{Zadatak: Paint}
\textsf{Pripremio: Tonko Sabolčec}\\
\textsf{Potrebno znanje: teorija grafova, \textit{union find} struktura podataka,
  metoda spajanja \textit{manji} na \textit{veći}}

Prvi podzadatak može se riješiti šetnjom po istobojnom susjedstvu koristeći DFS
ili BFS algoritam. Složenost jedne operacije bojanja u tom je slučaju linearno
ovisna o broju piksela u komponenti (istobojnom susjedstvu). U
najgorem slučaju komponenta će imati $O(R \cdot S)$ piksela, pa je ukupna složenost
takvog pristupa $O(R \cdot S \cdot Q)$.

Drugi podzadatak specijalni je slučaj kada se slika sastoji od samo jedne
dimenzije. Neke od ideja kako pristupiti tom problemu su:
\begin{itemize}[topsep=0pt]
  \item Održavanje skupa komponenti (kontinuirani intervali - npr. korištenjem
    \verb|std::set| u C++) koji omogućava brzo dohvaćanje susjeda i spajanje
    intervala (brisanje i dodavanje u skup).
  \item Održavanje povezanih komponenti pomoću \textit{union find} strukture
    podataka.
  \item Podjela niza na dijelove veličine $O(\sqrt{S})$ uz održavanje
    zastavice za svaki dio koja označava čini li taj dio jednu komponentu
    (tj. je li cijeli dio jednobojan?).
\end{itemize}

Treći podzadatak ima sljedeću posebnost: u pojedinom koraku ili neće biti 
nikakvih promjena (ako boja kojom je kantica napunjena odgovara boji komponente
koju bojimo) ili ćemo komponenti koju bojimo promijeniti boju i povezati
je sa \textit{svim} njezinim susjednim komponentama. Za održavanje komponenti
može se koristiti \textit{union find} struktura podataka, uz koju ćemo za svaku
komponentu održavati i skup (npr. \verb|std::set| u C++) aktivnih susjeda. Prilikom
prolaska po skupu susjeda trenutne komponente, tj. prilikom spajanja trenutne sa susjednom komponentom,
trenutnoj komponenti brišemo susjednu komponentu iz skupa (i obrnuto).
Spajanja neće biti više od $O(R \cdot S)$, a budući da svako spajanje ima
složenost operacije brisanja iz skupa, $O(\log (R \cdot S))$, ukupna je složenost
$O(R \cdot S \cdot \log (R \cdot S))$.

Na prvi pogled može djelovati da bi se algoritam za treći podzadatak mogao
koristiti i za konačno rješenje. No, problem je u tome što ne moramo spojiti
sve susjedne komponente u svakom koraku, a ipak moramo prolaziti po svim
susjednim komponentama da provjerimo boju. Možda bismo mogli doskočiti tom
problemu tako da održavamo susjede grupirane po bojama
(npr \verb|std::map| u C++), no prilikom promjene boje na
trenutnoj komponenti, opet bismo morali prolaziti po svim susjedima da ih se
obavijesti o promjeni boje trenutne komponente. Imajmo zasad na umu spomenute
opservacije, no krenimo ipak od rješenja prvog podzadatka. Primijetite da,
ako bi prosječna veličina komponenti koje bojimo bila dovoljno mala,
primjerice oko 10 piksela, tada bi algoritam za prvi podzadatak bio dovoljno
brz. Naime, ukupna složenost iznosila bi $O(R \cdot S + Q \cdot 10)$. No, što ako je
prosječna veličina komponenti veća? Za prosječnu veličinu komponenti 100
piksela složenost je i dalje dovoljno dobra. Za 1000 piksela vjerojatno
također. Za 10000 piksela? Teško. Međutim, primijetite da povećavanjem te granice
smanjujemo broj komponenti koji tu granicu nadmašuju. Drugim riječima, u
svakom trenutku postoji najviše $R \cdot S / K$ komponenti koje se sastoje od $K$ ili
više piksela (o odabiru konkretne vrijednosti $K$ bit će više riječi kasnije).
Ovo svojstvo moglo bi se nekako iskoristiti, intuicija je sljedeća: za
komponente koje se sastoje od $K$ ili manje piksela, primjenjujemo neki od
naivnih pristupa sličnih onom iz prvog podzadatka, dok za komponente s više
od $K$ piksela pristupamo na drukčiji način (koji ne bi bio dovoljno učinkovit
kad bismo ga primjenjivali na veći broj manjih komponenti).

Konkretno, za svaku komponentu pamtimo sljedeće vrijednosti: \textit{boju},
\textit{veličinu} i \textit{popis velikih susjednih komponenata} (s više od $K$
piksela -- primijetite da njih ima najviše $R \cdot S / K$). Prilikom spajanja
komponenti i održavanja spomenutih vrijednosti po običaju koristimo
\textit{union find} strukturu podataka. Dodatno, specijalno za \textit{velike}
komponente održavamo i \textit{popis susjeda grupiranih po boji}. Algoritam ćemo
objasniti pomoću sljedećih tri operacija:
\begin{enumerate}
  \item Operacija \textit{Oboji}$(x, y, boja)$ -- primjena kantice za ispunu
    na pikselu $(x, y)$:
    \begin{itemize}[topsep=0pt]
      \item Određuje se oznaka komponente $c$ kojoj pripada piksel $(x, y)$.
      \item Pristupa se operaciji \textit{SusjediUBoji}$(c, boja)$ za određivanje
        oznaka susjednih komponenti iste boje, $c'_1, c'_2, ..., c'_l$.
      \item Dobivene komponente $c'_i$ spajaju se s trenutnom komponentom $c$
        ($c = \textrm{\textit{Spoji}}(c, c'_i)$).
      \item Nakon spajanja, prolazi se po svim velikim susjedima dobivene
        komponente i \textit{obavještava} ih se o promjeni boje (tj. dodaje se
        oznaka trenutne komponente $c$ u popis susjedne komponente za zadanu boju).
    \end{itemize}

  \item Operacija \textit{SusjediUBoji}$(c, boja)$ -- pronalazak susjednih
    komponenti zadane boje:
    \begin{itemize}[topsep=0pt]
      \item Ako je komponenta $c$ mala, prolazi se po svim susjedima
        komponente (kojih ima $O(K)$) i za svakog provjeri koje je boje.
      \item Ako je komponenta $c$ velika, dostupan je popis susjeda po bojama,
        koji se u tom slučaju izravno koristi. \\
        \textbf{Zamjedba}: Nakon prolaska po popisu obojenih susjeda, taj se
        popis može (mora) obrisati (jer nakon spajanja više neće postojati susjed
        zadane boje -- sve će biti dio jedne komponente).
    \end{itemize}

  \item Operacija \textit{Spoji}$(c_1, c_2)$ - spajanje komponenti $c_1$ i $c_2$:
    \begin{itemize}[topsep=0pt]
      \item Veličina nastale komponente dobiva se zbrajanjem veličina komponenti
        $c_1$ i $c_2$.
      \item Udruženi popis velikih susjeda dobiva se metodom spajanja manjeg
        skupa na veći skup. (Moguće je da nakon ove operacije u skupu imamo
        oznake komponenti koje su u međuvremenu nestale, tj. spojene u neku
        veću komponentu. Za pristup tom problemu vidi implementacijski detalj
        niže u tekstu.)
      \item U slučaju spajanja dviju velikih komponenti, potrebno je stvoriti
        i udruženi popis susjednih komponenti po bojama. Pritom se također
        koristi metoda spajanja manjeg skupa na veći.
    \end{itemize}
\end{enumerate}

Operaciju \textit{Spoji} obavit ćemo najviše $O(R \cdot S)$ puta. Metoda spajanja
manje liste na veću osigurava ukupnu amortiziranu složenost $O(R \cdot S \cdot \log (R \cdot S))$.
Složenost svih operacija \textit{SusjediUBoji} u slučaju malih komponenti je
$O(R \cdot S \cdot K)$, dok je amortizirana složenost u slučaju velikih komponenti
$O(R \cdot S)$. Konačno, složenost svih operacija \textit{Oboji} je $O(Q \cdot R \cdot S / K)$. Ukupna
složenost ovisi o parametru $K$ i iznosi $O(R \cdot S \cdot \log(R \cdot S) + R \cdot S \cdot K + \frac{Q \cdot R \cdot S}{K})$.
Slijedi da se optimalna složenost dobiva za $K = O(\sqrt{Q})$,
koji bi za konkretna ograničenja u zadatku iznosio između 100 i 500.

Spomenimo i još jedan \textbf{implementacijski detalj}: Prilikom spajanja susjeda,
moguće je da će neke oznake u listi velikih susjeda i listi susjeda grupiranih
po bojama postati nevažeće ili da će se boja susjeda u međuvremenu promijeniti.
Nije potrebno osmišljavati metode kojima će se takve nevažeće komponente pronaći
i izbrisati iz popisa. Dovoljno je prilikom prolaska kroz popis dodatno provjeriti
trenutno stanje i usporediti ga s očekivanim, preskakujući oznake koje ne
zadovoljavaju očekivane uvjete.
