%%%%%%%%%%%%%%%%%%%%%%%%%%%%%%%%%%%%%%%%%%%%%%%%%%%%%%%%%%%%%%%%%%%%%%
% Problem statement
\begin{statement}[
  problempoints=100,
  timelimit=1 sekunda,
  memorylimit=512 MiB,
]{Paint}

Vratimo se najprije 35 godina u prošlost, točnije u 1985.\ godinu. Ta je godina
po mnogočemu posebna, Nintendo na američko tržište plasira prvu NES igraču
konzolu, Richard Stallman objavljuje \textit{GNU Manifesto}, na svijet dolaze
nogometne ikone poput Cristiana Ronalda, Luke Modrića i Darija Jerteca, a
gospodin Malnar za jedanaesti rođendan dobiva \textit{Windows 1.0} -- prvo
izdanje popularnog operacijskog sustava uz pomoć kojeg će godinu dana kasnije
otkriti rekurziju. No, zasad se bavi proučavanjem alata za ispunu (popularne
\textit{kantice}) u programu \textit{MS Paint} te mu na pamet pada ovaj zadatak.

Prostor za crtanje u programu \textit{MS Paint} zamišljamo kao pravokutnu
matricu koja se sastoji od $R$ redaka i $S$ stupaca. Svako polje te matrice
predstavlja jedan piksel koji može biti obojen u neku od $10^9$ boja koje
korisnik ima raspolaganju. Kada kanticu napunjenu bojom $A$ primijenimo na
piksel na polju $(r, s)$ obojen bojom $B$, tada svi pikseli
\textit{istobojnog susjedstva} polja $(r, s)$ postaju obojeni bojom $A$.
Istobojno susjedstvo polja $(r, s)$ je skup polja do kojih je moguće doći
šetnjom u četiri smjera (gore, dolje, lijevo i desno) od polja $(r, s)$ ne
mijenjajući boju piksela na putu. Primijetite da je i samo polje $(r, s)$ dio
svog istobojnog susjedstva.

\textbf{TODO: skice istobojnog susjedstva}

Zadana je početna slika u programu \textit{MS Paint} nad kojom je $Q$ puta
primijenjen alat za ispunu. Vaš je zadatak odrediti završno stanje slike.

%%%%%%%%%%%%%%%%%%%%%%%%%%%%%%%%%%%%%%%%%%%%%%%%%%%%%%%%%%%%%%%%%%%%%%
% Input
\subsection*{Ulazni podaci}
U prvom su retku prirodni brojevi $R$ i $S$ iz teksta zadatka.

U sljedećih je $R$ redaka po $S$ brojeva koji predstavljaju početnu sliku
u programu \textit{MS Paint}. Preciznije, $j$-ti broj $i$-tog retka slike
predstavlja boju piksela na polju $(i, j)$.

U sljedećem je retku prirodan broj $Q$ iz teksta zadatka.

U $i$-tom od sljedećih $Q$ redaka nalaze se prirodni brojevi $r_i$, $s_i$ i
$c_i$ $(1 \le r_i \le R, 1 \le s_i \le S, 1 \le c_i \le 10^9)$, koji
označavaju ($i$-tu) primjenu kantice napunjene bojom $c_i$ na polje $(r_i,
s_i)$.

%%%%%%%%%%%%%%%%%%%%%%%%%%%%%%%%%%%%%%%%%%%%%%%%%%%%%%%%%%%%%%%%%%%%%%
% Output
\subsection*{Izlazni podaci}
Ispišite završno stanje slike u istom formatu kakvim je početno stanje
zadano u ulazu.

%%%%%%%%%%%%%%%%%%%%%%%%%%%%%%%%%%%%%%%%%%%%%%%%%%%%%%%%%%%%%%%%%%%%%%
% Scoring
\subsection*{Bodovanje}
{\renewcommand{\arraystretch}{1.4}
  \setlength{\tabcolsep}{6pt}
  \begin{tabular}{ccl}
 Podzadatak & Broj bodova & Ograničenja \\ \midrule
  ?? & ?? &  \\
\end{tabular}}

%%%%%%%%%%%%%%%%%%%%%%%%%%%%%%%%%%%%%%%%%%%%%%%%%%%%%%%%%%%%%%%%%%%%%%
% Examples
\subsection*{Probni primjeri}
\begin{tabularx}{\textwidth}{X'X'X}
\sampleinputs{test/paint.dummy.in.1}{test/paint.dummy.out.1} &
\sampleinputs{test/paint.dummy.in.2}{test/paint.dummy.out.2} &
\sampleinputs{test/paint.dummy.in.3}{test/paint.dummy.out.3}
\end{tabularx}

\textbf{TODO: clarification}

%%%%%%%%%%%%%%%%%%%%%%%%%%%%%%%%%%%%%%%%%%%%%%%%%%%%%%%%%%%%%%%%%%%%%%
% We're done
\end{statement}

%%% Local Variables:
%%% mode: latex
%%% mode: flyspell
%%% ispell-local-dictionary: "croatian"
%%% TeX-master: "../hio.tex"
%%% End:
